\section{Related Work}
%This paper explores the relationship between explanations and feedback in ML. First 
We review related work in interactive and explainable ML, separately, and then describe prior studies on their relationship. 

%Interactive ML
\subsection{Interactive Machine Learning}

Compared to classical supervised ML's focus on static labels and
datasets, interactive machine learning (IML) trains a model through rapid end \emph{user}
interaction~\cite{Amershi2014PowerLearning}.
%
IML commonly produces higher quality models~\cite{Settles2010ActiveSurvey, Raghavan2006ActiveInstances}, personalized recommendations~\cite{Amershi2012Regroup:Networks, Gowacka2013DirectingKeywords} or
models that are better aligned with users'
understanding~\cite{Andrzejewski2009IncorporatingPriors,
Lee2012IVisClustering:Modeling}.
%
However, user feedback can have negative effects: decreased system 
performance~\cite{Ahn2007OpenHarm, Wu2019LocalAnalysis} or 
inconsistent mental models~\cite{Bansal2019UpdatesTradeoff}.

This ``human-in-the-loop'' approach has been applied across machine
learning, including in supervised
ML~\cite{Fails2003InteractiveLearning, Settles2010ActiveSurvey},
unsupervised ML~\cite{Balcan2008ClusteringFeedback}, and reinforcement
learning~\cite{Knox2012ReinforcementTasks,
Saunders2018TrialIntervention}.
%
We focus on supervised ML as it provides intuitive mechanisms for non-ML expert, end user
feedback, such as providing training
examples~\cite{Fiebrink2009ALearning},
preferences~\cite{Resnick1994GroupLens:Netnews}, or by
reacting to model predictions with instance-level (i.e., correcting or
confirming predictions~\cite{Fails2003InteractiveLearning,
Culotta2006CorrectiveExtraction}) or feature-level feedback (i.e.,
denoting features indicative of each
class~\cite{Settles2011ClosingInstances,
Kulesza2015PrinciplesLearning, Raghavan2006ActiveInstances}). Our focus on end users providing feedback, and in particular feature-level feedback aligns with ``Machine Teaching'', where non-ML experts build models from more than just labeled data~\cite{Wall2019UsingTeaching}.

%
%Our studies use a text classification system that supports instance
%and feature-level feedback, similar to
%Settles~\cite{Settles2011ClosingInstances} and Kulesza et
%al.~\cite{Kulesza2015PrinciplesLearning}.

\subsection{Explainable Machine Learning}
% Explainable Machine Learning
% in general
Explainability (or intelligibility) in ML has received growing
attention as ML models take on more important responsibilities in
society.
%
More complex models are often more accurate.
%
Thus, intelligibility research both develops global explanations, such as more transparent
models~\cite{Caruana2015IntelligibleReadmission,Alvarez-Melis2018TowardsNetworks,
Lage2018Human-in-the-loopPrior, Si2013LearningDetection} or black-box explanations~\cite{Lakkaraju2019FaithfulModels}, and local explanations of individual algorithm decisions, which can include
input evidence~\cite{Lei2016RationalizingPredictions,
Feng2019WhatPlay},
localizations~\cite{Selvaraju2017Grad-CAM:Localization,
Park2017AttentiveAbstract}, natural language
explanations~\cite{Camburu2018E-SNLI:Explanations,
Ehsan2019AutomatedPerceptions, Gkatzia2016NaturalInformation}, or
local approximations~\cite{Ribeiro2016WhyClassifier}.
%
We focus on local explanations (i.e., highlighting
important words).

Explanations can support
fairness and bias assessments~\cite{Dodge2019ExplainingJudgment}, improve perceived understanding~\cite{Kocielnik2019WillSystems},
promote system
acceptance~\cite{Herlocker2000ExplainingRecommendations}, engender
trust~\cite{Pu2006TrustInterfaces}, and convince users to accept
recommendations~\cite{Cramer2008TheRecommender}. However, explanations can decrease users' perceptions when algorithmic limitations or uncertainty
are portrayed~\cite{Cai2019TheInterface,
Stowers2017InsightsUncertainty, Lim2011InvestigatingApplications}.
%
Explanations can have other negative effects, such as over-reliance~\cite{Stumpf2016ExplanationsSystems} or inability to detect
mistakes~\cite{Poursabzi-Sangdeh2018ManipulatingInterpretability}.
%
ML-based systems can set expectations by exposing accuracy~\cite{Yin2019UnderstandingModels} or anticipated system mistakes~\cite{Kocielnik2019WillSystems}. This work explores whether such insight in turn increases users desire to fix mistakes and improve systems.
%
%This work explores how explanations with or without the opportunity
%for feedback impact user satisfaction (frustration, trust, acceptance)
%and how users expect these models to change.

Prior work explores the effect of explanations on mental models, in
particular on \textit{predictability}, or the users' ability to
predict model
behavior~\cite{Poursabzi-Sangdeh2018ManipulatingInterpretability,
Chandrasekaran2019DoHuman, Bunt2007UnderstandingCustomization},
finding conflicting results. Explanations improved predictability for apartment pricing~\cite{Poursabzi-Sangdeh2018ManipulatingInterpretability} and GUI customization~\cite{Bunt2007UnderstandingCustomization}, but did not have an effect for a visual question answering~\cite{Chandrasekaran2019DoHuman}.
%
This discrepancy could be because that users expected the ML model to change and therefore were less successful at predicting future model behavior.
%
Our studies measure expected change by asking
users whether they think the system they evaluated will have higher, similar, or lower accuracy on new data.

\subsection{Relationship of Explainability and Interactivity in ML}
% explainable + interactive

Users need to understand how models work~\cite{Fiebrink2009ALearning, Amershi2010ExaminingLearning, Kulesza2012TellAgent} to best fix
them, and how models are explained changes user
feedback~\cite{Rosenthal2010TowardsData,
Kulesza2015PrinciplesLearning}.
%
%Rosenthal and Dey's study~\cite{Rosenthal2010TowardsData} suggest that presenting a system's prediction and low-level context (e.g., an email has
%keywords ``A'' and ``B'') helps users give effective feedback.
%
Kulesza et al.~\cite{Kulesza2015PrinciplesLearning} introduced EluciDebug, based on the concept of
``explanatory debugging'', in which a classifier
explains binary predictions to users in the form of important input
words and proportion of the data labeled as each class.
%
Users in turn inform the classifier by correcting the prediction---\textit{instance feedback}---or saying which words are important for each
class---\textit{feature feedback}.
%
Users both better understood and corrected  EluciDebug's mistakes compared to a system without explanations or feature-level feedback.
%
%, yet for simplicity, we show and support interaction with only the
%word features within each example.
While these studies tell us that explanations foster better feedback,
prior work has not investigated how user perceptions---such as
frustration and trust---are shaped by the presence or (sometimes more
importantly) absence of IML feedback and explanations. Therefore, we address this using a similar data set, task, explanation, and feedback mechanisms.




