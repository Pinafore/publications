\section{Introduction}
Complex machine learning (ML) models can be incomprehensible for end users who are not ML experts. While a model may have high accuracy on held-out test sets, users may also want to know {\em why} the model is making its predictions; if the model is right for the right reasons, they can be more confident that it will generalize or is operating without bias~\cite{Dodge2019ExplainingJudgment}. Automatic model explanations---such as ``why'' and ``why not'' justifications~\cite{Lim2009WhySystems} and feature visualizations~\cite{Kulesza2015PrinciplesLearning}---can provide intuition and increase user confidence and trust~\cite{Pu2006TrustInterfaces, Bunt2007UnderstandingCustomization}, human task performance~\cite{Stowers2017InsightsUncertainty, Feng2019WhatPlay,schmidt-aaai19},  satisfaction~\cite{Biran2017Human-centricPredictions}, and system acceptance~\cite{Herlocker2000ExplainingRecommendations}. Ongoing government research programs~\cite{Gunning2016ExplainableXAI}, focused academic conferences,\footnote{ACM Conference on Fairness, Accountability, and Transparency (https://fatconference.org/)} and recent legislation on the ``right to explanation''~\cite{Goodman2017EuropeanExplanation} have also fueled a general push for ML system transparency. 

Explanations are not an unmitigated good, however. Complex explanations may promote over-reliance when they are convincing~\cite{Stumpf2016ExplanationsSystems} or lower user satisfaction when they are confusing~\cite{Narayanan2018HowExplanation}. Explanations that expose system uncertainty or algorithmic limitations may negatively affect users' perceptions~\cite{Lim2011InvestigatingApplications, Stowers2017InsightsUncertainty, Cai2019TheInterface}, and users may ignore explanations entirely if the benefit to attending to them is unclear~\cite{Kulesza2013TooModels}. 

This paper investigates two additional complications of
explanations. %in \emph{interactive} settings where users have 
%when users are given varied (or no) support for providing
%feedback to systems. 
% RQ1 How does varying explanation (yes or no) and support for feedback (none, instance-level, feature-level) affect user experience; in particular, how does explanation without feedback affect frustration, trust, and acceptance?
First, if explanations increase users'
understanding of ML models and the errors they make, can
this insight in turn increase users' desires to ``fix'' them and therefore reduce satisfaction if they cannot?
%
Interactive ML allows user feedback, which \emph{can} improve
model accuracy~\cite{Fails2003InteractiveLearning,
Raghavan2006ActiveInstances, Settles2011ClosingInstances}, but not
always~\cite{Ahn2007OpenHarm, Wu2019LocalAnalysis}.
%
Here, explanations can help: by improving mental models, explanations
can improve user
feedback~\cite{Kulesza2015PrinciplesLearning,
Rosenthal2010TowardsData}. Still, researchers have only begun to
examine the relationship between explainability and interactivity in ML.
%
%The question of whether explanations impact the user's desire to
%provide feedback has not been explored.
% RQ2 Do users that do not provide feedback, still expect the model to improve (expectation of improvement)? And is this affected by whether they receive an explanation?

Second, intuitively, users who provide feedback should expect model improvement, but what about those who do not give feedback; might
explanations also cause users to expect model improvement over time?
%
Humans expect that others who are capable of explaining their mistakes will self-reflect and learn from those mistakes~\cite{Siegler2009MicrogeneticSelf-explanation}.
Explanations reveal why the model was incorrect for
particular instances, so ML novices, in particular, may similarly expect introspective behavior and learning from the experience, even without user feedback. 

%This paper investigates two questions about the \textit{relationship} explanations and user feedback.  When a user sees explanations: (1) are they frustrated when they cannot provide feedback (2) do they expect the model to improve. 

%method
To study how explanations and supports for user feedback affect users' 
experience with a ML model, we conducted two crowdsourced experiments
with $180$ participants each.
%
Both experiments are a common classification task: is a message about
``hockey'' or ``baseball''~\cite{Settles2011ClosingInstances, Kulesza2015PrinciplesLearning}.
%
Because we expected explanations and feedback would be particularly salient when the model could be improved, the first experiment used a \textit{lower quality} model~($\sim 75\%$ accuracy), trained on a handful of
training documents.
%
Participants reviewed
predictions made by the classification model with or
without \textit{explanations},
and with one of three levels of user \textit{feedback} to the
model: none, instance-level (correcting or confirming the model's prediction), and feature-level (telling the model \textit{how} to predict). 
%
We measured participants' subjective post-task satisfaction, including frustration and trust, as well as
how they expected the model to change. 
%
The second study experiment was exactly the same as the first, but with a \textit{higher quality} model ($\sim 95\%$ accuracy) to understand the effects of model quality.

%contributions
Our findings contribute the following observations to the nascent
understanding of interactive and explainable machine learning:
%(1) differing model quality results in different conclusions for impacts of explanations and feedback on user perceptions, as expected;
%
(1) users wanted the \textit{opportunity} to provide feedback, regardless of model quality or whether they received explanations; 
%
(2) for the low-quality model, feedback reduced frustration and increased trust and acceptance, but explanations had the opposite effect; therefore, explanations without the opportunity for feedback resulted in an especially negative user experience;
%
(3) for the high-quality model, users were not as frustrated, yet requesting feature-level feedback without an explanation reduced trust; %, and as with lower quality models, explanations without the opportunity for feedback may decrease satisfaction.
%
(4) regardless of model quality, when users  %told the model \textit{how} to work
provided detailed feedback, they expected more improvement; yet, users generally expected model improvement even for conditions without any user feedback, demonstrating possible misconceptions of ML models by end users.
%

Despite the constrained setting (i.e., a classical, binary text classification task, with a simple explanation), we see this work as an important step in illustrating a key relationship between explanations and feedback. We conclude this paper by discussing extensions to more complex tasks and models with more sophisticated explanation and feedback mechanisms.