Large Language Models (LLMs) are increasingly being deployed in interactive contexts that involve direct user engagement, such as chatbots and writing assistants. These deployments are increasingly plagued by prompt injection and jailbreaking (collectively, prompt hacking), in which models are manipulated to ignore their original instructions and instead follow potentially malicious ones. Although widely acknowledged as a significant security threat, there is a dearth of large-scale resources and quantitative studies on prompt hacking. 
% 
To address this lacuna, we launch a global prompt hacking competition, which allows for free-form human input attacks. We elicit 600K+ adversarial prompts against three state-of-the-art LLMs.
%
We describe the dataset, 
which empirically verifies that current LLMs can indeed be manipulated via prompt hacking. We also present a comprehensive taxonomical ontology of the types of adversarial prompts.
%
% We hope our dataset and analysis not only enable the community to better understand the vulnerabilities of LLMs, but also inspire future work on potential mitigation strategies. 
%
% We openly release all collected data to facilitate future research. 