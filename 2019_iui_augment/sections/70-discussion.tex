\section{Discussion}
\label{sec:discussion}

% \fscomment{Need to explain in more detail, but the main takeaways are
%
% \begin{itemize}
%     \item our evaluation is a good extrinsic and application-grounded
%         evaluation
%     \item the effectiveness of interpretations varies with different
%         inputs and human users of different skill levels.  this
%         suggests the possibility of making the interpretations more
%         useful by automatically selecting ones to show based on the
%         input, and adapting to specific human user. RL or bandit
%     \item when computer performance is higher than human,
%         interpretation should pay more attention to showing
%         uncertainty
% \end{itemize}
% }

% \jbgcomment{I've been moving stuff here that delayed the point in
%   previous sections.  It's not well integrated or organized}
%
% \jbgcomment{Should have explicit design recommendations for people
%   building human in the loop interfaces}

% \subsection{Experts vs.\ Novices}
% In our cooperative setting, experts and novices use model
% interpretations differently.  Novices tend to be too trusting, playing
% much more aggressively when visualizations are enabled
% (Figure~\ref{fig:tools_buzz}).
% Interpretations that require less processing appear more useful to
% experts (Figure~\ref{fig:coefs}).
% For tasks where computer performs on the same level as human experts
% but shows different strengths,
% interpretations should be intuitive, easy to
% process.
% An unstructured saliency mapping
% over input features might not be effective enough
% (Figure~\ref{fig:fail_to_convince}).

The evaluation we present is grounded in a realistic setting, but also
task-specific. This section discusses how our method can be directly
applied to other settings, its limitations, and how we can incorporate
other components such as an eye tracker to our framework for a more
fine-grained assessment of interpretability.

\subsection{Forms and Methods of Interpretation}

Interpretations take on many forms, and within each form we have multiple
methods to generate the interpretation. For example, to highlight
salient input features for image classification, we can use variants
of input gradient~\cite{baehrens2010explain, simonyan2013deep}.
To optimize the generalizability of our results (despite being
task-specific) and demonstrate the flexibility of our method, we
focus on a comparison between forms of interpretation. To
select one method of each form, we choose a
high-performance linear model for its canonical interpretations.  Our
evaluation framework, including the interface and the regression
analysis, can be directly applied to a different comparison---one
between multiple methods of the same form.  This comparison is
particularly useful in the case of neural models, where all existing
interpretations are some approximation, and the evaluation of
how faithful they are to the model is crucial.


\subsection{Intrinsic and Extrinsic Evaluation}

Our approach is an extrinsic evaluation~\cite{narayanan2018humans}. The
task is played by thousands who compete in regularly.  Using
\qb{} allows a contextual, motivated evaluation of whether an
interpretation is useful. In contrast, intrinsic evaluation relies on
the interpretation alone. It is more direct but limited. In tasks
where no ground-truth explanation is available, the most tractable and
commonly used method is to construct ground-truth using a simpler
model as a benchmark for interpretability. For example, weights of linear models are
used for evaluating input highlight
explanations~\cite{li2016understanding,murdoch2018cd}.
This is restricted to tasks where the benchmark model performs
similarly to the complex model that requires interpretation, and it
does not work in application-grounded setting
(Section~\ref{sec:eval_qb}).

Extrinsic evaluations are hard to design, as they are affected by more
factors, especially humans' trust. When a user does not trust the
model and ignores it, the difference in the performance is not
affected by the explanations at all. Narayanan~\etal{}~\cite{narayanan2018humans} uses
``alien'' tasks to enforce trust, tasks that humans do not have
knowledge of. Our approach, in contrast, considers trust as an
inherent part of the cooperation: good interpretations should be
consistent and intuitive to convince humans to use it.

% generalizability
% 1. comparison between underlying algorithms
% 2. other tasks
% This paper focuses on comparing forms of
% interpretations, so we limited the experiment to one method in each
% form. However, our evaluation framework, the interface, and the
% experimental setup can be naturally adapted to other comparisons, for
% example, between different underlying algorithms within each
% form, or between different models with interpretation methods fixed.

\subsection{Generalizing to Other Tasks}
Our method can be applied to natural language tasks other than
\qb{}, although \qb{}'s characters make it uniquely suitable.  To use
our interface for some other text classification task, for example
sentiment analysis or spam detection, one can convert the task into an
incremental version where the input is shown
word-by-word. Time limitation or competition can be added to
encourage the users to pay attention to
visualizations~\cite{narayanan2018humans}. One task
related to \qb{} has wide real world application: simultaneous
interpretation (or simultaneous translation, not to be
confused with model interpretation). Interpreters need to trade off
between accuracy and delay, much like \qb{}ers need to balance
accuracy and aggressiveness. The underlying mechanism of the
\abr{qanta} buzzer~\cite{he2016opponent} also resembles how
simultaneous translation systems handle this
trade-off~\cite{Grissom-14}.

% limitations
% 1. UI design, equal exposure (placement)
% 2. ordering of questions, learning curve
\subsection{Limitations}
First, because we
compare visualizations individually and in combinations, their
placement is fixed to avoid confusing the players. The fixed
placement leads to uneven exposure to the users, so they might pay
less attention to some visualizations than others. If we focus on
individual visualizations, one way to resolve this issue is to display
the interpretation in a single fixed location, for example below the
question area. This would lead to a fair display of different
visualizations without confusing the users. However, one single
location might not suit all visualizations: for example, input
highlight should collocate with the input, while evidence is best
displayed next to the input for comparison.

Visualizations displayed on our interface change from question to
question, and the randomization (Setup)
might confuse the users. Before answering questions, each user sees
a tutorial that walks through the components of the interface,
but this can be improved by a set of warm-up questions to familiarize
the users of the interaction, which we will implement in future
studies. In addition, we can randomly sort the questions instead of
the visualizations, so the users see the same layout for multiple
questions, reducing context switches and consequently the cognitive
load.

Another limitation of our study is that, when a player's performance
improves with some interpretation, we cannot tell how much of that
improvement comes from the player using that interpretation.
We cannot derive causality from correlation. The key
missing factor is how much attention the player gave the interpretation,
and how much the decision is based on that. The attention the player
gave each interpretation could be measured using an eye tracker, and we
leave this to future work.


% \subsection{A Question of Trust}
%
% However, we believe that trust should not be separated from the
% evaluation, as it is an inherent part of a human-in-the-loop
% system. There will always be a varying degree of trust in the system
% when it is used in the real world. And to keep the end-to-end
% assessment consistent with the real world, we should not force trust
% in our extrinsic evaluations. Moreover, it would be ideal if we can
% simlutate different degrees of human trust to make the evaluations
% more realistic and accurate.
%
% Trust can be affected by many factors: accuracy of the system,
% consistency of the explanation, and also external rewards. Reward
% structures of competitive environments can incentivize humans to use
% the model outputs in a certain way.  For example, when a large
% negative reward is given on incorrect prediction, the users would be
% encouraged to be more careful and less reliant on the model. By tuning
% the external rewards, we can achieve more fine-grained control,
% without resorting to ``alien" tasks.
%
% One benefit of our framework as a competition is that we can adjust
% players' loss function through the \qb{} scoring mechanism.  For
% example, we can offer ``power'' points for early answers (typically in
% competition) or increase the penalty for getting questions wrong
% (sometimes seen with larger player pools).  This can encourage players
% to depend on the visualization more (increasing rewards) or to be more
% skeptical (increasing penalties).

%% \subsection{Trust Factor and Threat to Validity}
%% As mentioned in Section~\ref{sec:eval_qb}, degree of trust can be adjusted by
%% tuning the reward structure, but we cannot guarantee consistent trust across
%% player. The main threat to the validity of our results is the variance in player
%% baseline performance, difficulty of questions, and degree of trust. We plan to
%% collect more data, which would allow the regression analysis to factor out these
%% variance more accurately.

% \subsection{Persuasive vs. Informative}

\subsection{Future Work}

While we focus on broad categories of interpretations to reveal that
some visualizations are more effective than others (e.g., highlighting
is more useful than guess lists), we can also use this approach to
evaluate specific highlighting methods in a task-based setting.  This
can help reveal how best to choose spans for highlighting, which words are best
suited for highlighting, and how to convey uncertainty in
highlighting.

While our evaluation focuses on the downstream task, we can expand
our analysis to measure how much users look at visualizations and in
what contexts (e.g., with an eye tracker).  This would reveal situational usefulness of
visualization components; if, for example, highlighting were only
useful to distinguish when two guesses had similar scores, we could
decrease cognitive load by only showing highlights when needed.

A tantalizing extension is to make these modifications automatically,
using the reward of task performance to encourage a reinforcement
learning algorithm to adjust interface elements to optimize
performance: such as changing font sizes, setting buttons for users to
explicitly agree or disagree with model predictions, or modifying the
highlighting strategy.

% \subsection{Other Tasks}
%
% Another possible extension is to convert any language task into a
% competitive version for interpretation evaluation. Natural language is
% inherently sequential, so converting the tasks into incremental
% versions is trivial. Then we can turn them into competitive games. For
% example, predicting the verb in simultaneous translation is very
% similar to the \qb{} answering task~\cite{Grissom-14}.  By comparing
% the evaluation in both incremental and non-incremental settings, we
% might be able to quantitatively determine whether the competitiveness
% achieved by this conversion indeed lead to more realistic and accurate
% evaluation.
%
% Previous work have studied the evaluation of interpretation in
% cooperative~\ref{Antunes2008StructuringDF}, and
% application-grounded~\ref{williams2016axis} settings. However, our
% work is the first to conduct application-grounded evaluation in
% natural language processing domain.
