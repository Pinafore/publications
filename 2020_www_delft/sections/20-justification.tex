\section{Task Definition}
\label{sec:justification}

The problem we consider is general \textit{factoid question
  answering}~\cite[inter alia]{iyyer2014neural, berant2013semantic, bordes2015large, 
  cai2013large}, in which the
system answers natural language questions using facts, i.e., entities
or attributes from knowledge graphs.
%
Factoid \abr{qa} is both widely studied in academia and a practical
tool used by commercial search engines and conversational assistants.
However, there is a discrepancy between the widely studied factoid
\abr{qa} academic benchmarks and  real applications:
the benchmarks are often designed for the existing relations in
knowledge graphs~\cite{berant2013semantic,bordes2015large}, while in reality,
minimal \abr{kg} coverage preculdes broader deployment.

For example, WebQuestions~\cite{berant2013semantic} are
written by crowdworkers targeting a triple in Freebase; \abr{l}{\small
  c}-quad~\cite{dubey2019lc} targets Wikidata and \dbpedia{}: the
questions are guaranteed to be covered by the \abr{kg} relations.
%
However, although the entities in your favorite
\abr{kg} are rich, the recall of closed-form relations is often
limited.
%
For example, in Figure~\ref{fig:graph}, the answer \underline{Delft}
is an entity in Wikipedia and Freebase, but the relation ``painting
the view of'' appears in neither DBpedia nor Freebase and is unlikely
to be covered in other knowledge graphs.
%
The relationships between complicated, multi-faceted entities defy
trite categorization in closed-form tuples; depending on a knowledge
graph to provide all the possible relationships limits the potential
of \abr{kgqa} systems.

We focus on a more realistic setting: open domain factoid
question answering, whose questions are designed to test \emph{human}
knowledge and include ineffable links between concepts~\cite{jennings-06}.
%
The datasets in our evaluation are solvable by humans using knowledge
about real world entities; \name{} should also be able to find the
answer entity without relying on explicit closed form relations.

Figure~\ref{fig:graph} shows an example question. 
The question has multiple \leftnode{}s (left side of graph), 
and links the question entities to \rightnode{}s (right). 
Intuitively, to find the answer, the system needs to first extract
multiple clues from different pieces of question.
%
Our proposed \name{} constructs a high coverage Free-Text Knowledge
Graph by harvesting natural language sentences in the corpus as graph
edges and grounds each question into its related subgraph
(Section~\ref{sec:evidence}).
%
Then, to model over the fruitful but noisy graph, it uses a \abr{gnn}
to distinguish useful evidence from noise and then aggregates them to
make the prediction (Section~\ref{sec:model}).

