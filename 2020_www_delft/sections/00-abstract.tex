\begin{abstract}


We introduce \name{}, a factoid question answering system which
combines the nuance and depth of knowledge graph
question answering approaches with the broader coverage of free-text.
%
\name{} builds a free-text knowledge graph from Wikipedia, 
with entities as nodes and sentences in which entities co-occur as edges.
%
For each question, \name{} finds the subgraph linking question entity
nodes to candidates using text sentences as edges, creating a dense
and high coverage semantic graph.
% 
A novel graph neural network reasons over the free-text graph---combining
evidence on the nodes via information along edge
sentences---to select a final
answer.
% 
Experiments on three question answering datasets show \name{} can
answer entity-rich questions better than machine reading based
models, \abr{bert}-based answer ranking and memory networks.
% We discuss and demonstrate two sources of effectiveness in \name{}.
%CZ rewrite on Oct 13
\name{}'s advantage comes from both the high coverage of its free-text
knowledge graph---more than double that of \dbpedia{} relations---and
the novel graph neural network which reasons on the rich but 
noisy free-text evidence.

\end{abstract} 
