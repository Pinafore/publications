\section{The View Beyond \name{}}
\label{sec:conclu}

Real-world factoid \abr{qa} requires answering diverse questions across domains.
%
Relying on existing knowledge graph relations to answer these questions often leads to highly accurate but brittle systems:
they suffer from low coverage.
%
To overcome the bottleneck of structure sparsity in existing knowledge graphs,
% solve the sparse 
% coverage issue, 
\name{} inherits 
\abr{kgqa}-style reasoning with the widely available free-text evidence.
\name{} builds a high coverage and dense free-text knowledge graph, using
natural language sentences as edges.
%
To answer questions, \name{} grounds the question into the related
subgraph connecting entities with free-text graph edges and then uses
a graph neural network to represent, reason, and select the answer
using evidence from both free-text and the graph structure.

Combining natural language and knowledge-rich graphs is a common
problem:
%
e-mail and contact lists,
%
semantic ontologies and sense disambiguation,
%
and semantic parsing.
%
Future work should explore whether these approachs are also useful for
dialog, language modeling, or \textit{ad hoc} search.

More directly for question answering, more fine-grained reasoning
could help solve the example of Table~\ref{tab:example}:
%
while both Odysseus and Penelope have Telemachus as a son, only
Odysseus made a long journey and should thus be the answer.
%
Recognizing specific properties of nodes either in a traditional
\abr{kg} or in free text could resolve these issues.

