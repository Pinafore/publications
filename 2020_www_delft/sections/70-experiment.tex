\section{Experiments}
\label{sec:exp}

We evaluate on three datasets with expert-authored (as opposed to crowd-worker) questions.
%
\qblink~\cite{Elgohary:Zhao:Boyd-Graber-2018} is an entity-centric dataset with human-authored questions. 
The task is to answer the entity the question describes. We use the released dataset for evaluation.
%
\qanta~\cite{iyyer2014neural} is a \abr{qa} dataset collected from \qb{} competitions. 
Each question is a sequence of sentences providing increased information about the answer entity. 
%
\triviaqa~\cite{joshi2017triviaqa} includes questions from trivia
games and is a benchmark dataset for \abr{mr}. We use its unfiltered
version, evaluate on its validation set, and split 10\% from its
unfiltered training set for model selection.
%
Unlike the other datasets, \triviaqa{} is relatively simpler; it
mentions fewer entities per question; as a result \name{} has lower
accuracy.


\begin{table}[t]
  \centering
  %\small
  \begin{tabular}{lrrr}
    %\toprule
      & \textbf{\qblink{}} & \textbf{\qanta{}} & \textbf{\triviaqa{}}  \\ \toprule
    Training &$42219$ & $31489$ & $41448$\\ 
    Dev &$3276$ & $2211$ & $4620$\\ 
    Test &$5984$ & $4089$ & $5970$\\  \midrule
    \# Tokens  &$31.7 \pm9.4$ & $129.2\pm 32.0$ & $16.5\pm 8.6$ \\
    \# Entities &$6.8\pm2.4$ & $21.2\pm7.3$ & $2.2\pm1.3$\\ \midrule
    \% 1-3 Entities & $9.6\%$& $0$& $86.9\%$\\
    \% 4-6 Entities & $36.7\%$& $0$& $13.1\%$\\
    \% 7-9 Entities & $36.5\%$& $0$& $0$\\
    \% 10+ Entities & $17.1\%$& $100\%$& $0$ \\
  \bottomrule
  \end{tabular}
  \caption{The three expert-authored datasets used for experiments.
    All are rich in entities, but the \qanta{} dataset especially
    frames questions via an answer's relationship with entities
    mentioned in the question.}
  \label{tab:data}
  
\end{table}


\begin{table}[t!]
    \centering
    \small
    \begin{tabular}{lp{4cm}}
      \textbf{Layer}   & \textbf{Description} \\ \toprule
        All \abr{rnn} & 1 layer Bi-\abr{gru}, 300 hidden dimension \\
        All \abr{ffn} & 600 dimension, ReLU activation \\
        \abr{mlp} & 2 layers with 600, 300 dimenstions, ReLU activation \\
        Attention & 600 dimension Bilinear  \\
        Self-Attention & 600 dimension Linear \\
        Layers & $L=3$ \\
        \bottomrule
    \end{tabular}
    \caption{Parameters in \name{}'s \abr{gnn}.}
    \label{tab:para}
\end{table}

 

We focus on questions that are answerable by Wikipedia entities.  To
adapt \triviaqa{} into this factoid setting, we filter out all
questions that do not have Wikipedia title as answer. We keep $~70\%$
of the questions, showing good coverage of Wikipedia Entities
in questions.
%
All \qblink{} and \qanta{} questions have entities tagged by TagMe.
%
TagMe finds no entities in 11\% of \triviaqa{} questions; we further
exclude these.
%
Table~\ref{tab:data} shows the statistics of these three datasets 
sand the fraction of questions with entities.

\subsection{Question Answering Methods}
We compare the following methods:
\begin{itemize*}
\item \quest{}~\cite{Lu:2019:ACQ} is an unsupervised factoid \abr{qa}
  system over text. For fairness, instead of Google results we apply \quest{} on \abr{ir}-retrieved 
Wikipedia documents.

\item \drqa{}~\cite{chen2017reading} is a machine reading model for open-domain 
  \abr{qa} that retrieves documents and extracts answers.
  
\item \docqa{}~\cite{clark2018simple} improves multi-paragraph machine reading, 
and is among the strongest on \triviaqa{}. Their suggested settings 
and pre-trained model on \triviaqa{} are used.

\item \bertet{} fine-tunes \bert{}~\cite{devlin2018bert} on the question-entity name pair to rank candidate entities in \name{}'s  graph.
  
\item \bertsent{} fine-tunes \bert{} on the question-entity gloss sequence pair to rank candidate entities in \name{}'s  graph.
  
\item \memnn{} is a memory network~\cite{weston2014memory} using fine-tuned \bert{}.
It uses the same evidence as \name{} but collapses the 
graph structure (i.e., edge evidence sentences) by concatenating all
evidence sentences into a memory cell.
%While we use this model for \rightnode{} filtering, it can also select
%an answer directly.

\item We evaluate our method, \name{}, with \glove{} embedding~\cite{pennington2014glove} and \bert{} embeddings.
\end{itemize*}



\begin{table*}[t]
  \centering
  %\small
  \begin{tabular}{lrrr}
      & \textbf{\qblink{}} & \textbf{\qanta{}} & \textbf{\triviaqa{}}  \\ \toprule
    \# Candidate Answer Entities per Question &$1607\pm 504$ & $1857 \pm 489$ & $1533\pm 934$\\ 
    Answer Recall in All Candidates
    % Entity extraction recall 
    & $92.4\%$&$92.6\%$  &$91.5\%$\\
    Answer Recall after Filtering
    % Entity filtering recall 
    & $87.6\%$& $83.9\%$ &$86.4\%$ \\ 
    Answer Recall within Two Hops along DBpedia Graph* & 38\% & -- & -- \\
    \midrule
    \# Edges to Correct Answer Node (+) & $5.07\pm2.17$&$12.33\pm5.59$ &$1.87\pm 1.12$\\
    \# Edges to Candidate Entity Node (-) & $2.35\pm0.99$ & $4.41\pm2.02$& $1.21\pm0.35$ \\ 
    \# Evidence Sentences per Edge (+) & $12.3\pm11.1$& $8.83\pm6.17$ & $15.53\pm17.52$ \\ 
    \# Evidence Sentences per Edge (-) &$4.67\pm 3.14$& $4.48\pm1.88$&$3.96\pm3.33$\\
    \bottomrule
  \end{tabular}
  \caption{Coverage and density of generated free-text entity graph. 
  (+) and (-) mark the statistics on correct answer nodes and incorrect nodes, respectively.
  (*) is the result from our manual labeling on 50 \qblink{} questions.
  \label{tab:coverage}}
  
\end{table*}


\subsection{Implementation}


\begin{table*}[t]
  \centering
  \begin{tabular}{lrrrrr gg rrr}
      & \multicolumn{5}{c}{\textbf{\qblink{}}} &
                                                 \multicolumn{2}{c}{ \textbf{\qanta{}}} & \multicolumn{3}{c}{\textbf{\triviaqa{}}}  \\ 
      & ALL
      & 1-3 & 4-6 & 7-9 & 10+ 
      & ALL
      & 10+
      & ALL
      & 1-3  & 4-6   \\ \toprule
    \quest{} & 0.07 & 0 & 0.09 & 0.09& 0& -&- &- &-&-\\ 
    \drqa{} & 38.3 &  35.4 & 37.5 & 39.2 & 39.6
    &  47.4 & 47.4
    &  40.3 &40.1 & 41.2 \\ 
     \docqa{} & - &- & - &- &-&-
    & - 
    & 49.4 & 49.3 & 49.8 \\ \hline
    \bertet{} & 16.2  & 16.1& 16.3&16.2 &16.4
    &  34.2 & 34.2
     & 25.1 &24.5 & 29.0 \\ 
    \bertsent{} & 34.8  & 34.7& 34.8& 34.7&34.6
    & 54.2 & 54.2
    & 44.5 &44.4 &45.1  \\  
    \memnn{} & 35.8 & 32.7 & 36.1 & 36.5& 34.3
    &  56.1 & 56.1
    & 51.3 & 50.9& 54.0\\
    \midrule
    \name{}-\glove{} & 54.2 & 45.5 & 55.0 & 56.4 & 53.2
    &   65.8 & 65.8
    &  51.3 & 50.1 & 59.5 \\ 
    \name{}-\bert{}  
    & \textbf{55.1} & \textbf{46.8} & \textbf{55.5} & \textbf{57.1} & \textbf{55.5}
    &   \textbf{66.2}  & \textbf{66.2}
    & \textbf{52.0}  &\textbf{ 50.5} & \textbf{61.1} \\
    \bottomrule
  \end{tabular}
  \caption{Answer Accuracy (Exact Match) on ALL questions as well as
    question groups with different numbers of entities: e.g., 0--3 are
    questions with fewer than four entities.  We omit empty ranges
    (such as very long, entity-rich \qb{} questions).  \name{} has
    higher accuracy than baselines, particularly on questions with
    more entities.}

  \label{tab:results}

\end{table*}







Our implementation uses PyTorch~\cite{paszke2017automatic} and its \abr{dgl} \abr{gnn} library.\footnote{\url{https://github.com/dmlc/dgl}}
%
We keep top twenty candidate entity nodes in the training and fifty for testing; the top five sentences for each edge is kept.
%
The parameters of \name{}'s \abr{gnn} layers are listed in Table~\ref{tab:para}. For \name{}-\bert{}, we use \bert{} 
output as contextualized embeddings. 

For \bertet{} and \bertsent{}, we concatenate the question and entity
name (entity gloss for \bertsent{}) as the \bert{} input and
 apply an affine layer and sigmoid activation to the last \bert{} layer of the \abr{[cls]} token; the model
outputs a scalar relevance score.


\memnn{} concatenates all evidence sentences and the node gloss, and combines with the question
as the input of \bert{}.  Like \bertet{} and \bertsent{}, an affine layer and sigmoid activation is
applied on \bert{} output to produces the answer score.


\drqa{} retrieves $10$ documents and then $10$ paragraphs from them; we use the default 
setting for training. During inference, we apply TagMe to 
each retrieved paragraph and 
limit the candidate spans as tagged entities. 
\docqa{} uses the pre-trained model on \triviaqa{}-unfiltered
dataset with default configuration applied to our subset.
