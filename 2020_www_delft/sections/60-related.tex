\section{Related Work: Knowledge Representation for QA}
\label{sec:bg}

\name{} is impossible without the insights of 
traditional knowledge bases for question answering and question
answering from natural language, which we combine using graph neural
networks.
%
This section describes how we build on these subfields.

\subsection{Knowledge Graph Question Answering}

With knowledge graphs (\abr{kg}) like
Freebase~\cite{bollacker2008freebase} and
DBpedia~\cite{mendes2011dbpedia} enable question answering using their
rich, dependable structure.
%
This has spurred \abr{kg}-specific \abr{qa} datasets on general domain
large scale knowledge graphs: WebQuestions~\cite{berant2013semantic},
SimpleQuestions~\cite{bordes2015large}, and special-domain \abr{kg}s,
such as WikiMovies~\cite{miller2016key}.
%
In turn, these new datasets have prompted special-purpose 
\abr{kgqa} algorithms.
%
Some convert questions to semantic parsing problems and execute the
logical forms on the graph~\cite{cai2013large, kwiatkowski2013scaling,
  reddy2014large, yih2015semantic}.
%
Others use information extraction
 to first extract question related information in \abr{kg}
and then find the answer~\cite{bao2014knowledge, yao2014information,
  gardner2017open}.

These work well on questions tailored for the underlying \abr{kg}.
%
For example,
WebQuestions guarantee its questions can be answered by
Freebase~\cite{berant2013semantic}.
%
Though modern knowledge graphs have good coverage on
entities~\cite{cxthesis}, adding relations takes time and
money~\cite{Paulheim2018HowMI}, often requiring human
effort~\cite{bollacker2008freebase} or scraping human-edited
structured resources~\cite{mendes2011dbpedia}.
%
These lacun\ae represent impede broader use and
adoption.

Like \name{}, \abr{quest}~\cite{Lu:2019:ACQ} seeks to address this by
building a noisy quasi-\abr{kg} with nodes and edges, consisting of
dynamically retrieved entity names and relational phrases from raw
text.
%
Unlike \name{}, this graph is built using existing Open Information
Extraction (\abr{ie}).
%
Then it answers questions on the extracted graph.
%
Unlike \name{}, which is geared toward recall, \abr{ie} errs toward
precision and require regular, clean text.
%
In contrast, many real-world factoid questions contain linguistically
rich structures, making relation extraction challenging.
%
We instead directly extract free-text sentences as indirect relations
between entities, which ensures high coverage of evidence information
to the question.


Similarly, {\abr{graft}-\abr{n}\small et}~\cite{sun2018open} extends an existing \abr{kg} with 
text information.
%
It grafts text evidence onto \abr{kg} nodes but retains the original \abr{kg} relations.
%
It then reasons over this graph to answer \abr{kg}-specific questions.
%
\name{}, in contrast, grafts text evidence onto both nodes and edges
to enrich the relationships between nodes, building on
the success of unconstrained ``machine reading'' \abr{qa} systems.

\subsection{Question Answering over Text}

Compared with highly structured \abr{kg}, unstructured text
collections (e.g., Wikipedia, newswire, or Web scrapes) is cheaper but
noisier for \abr{qa}~\cite{chen2018neural}.
%
Recent datasets such as \squad{}~\cite{rajpurkar2016squad},
\triviaqa{}~\cite{joshi2017triviaqa}, \abr{ms marco}~\cite{marco} and
natural questions~\cite{kwiatkowski2019natural} are typically solved
via a coarse search for a passage (if the passage isn't given) and
then finding a fine-grained span to answer the question.

A rich vein of neural readers match the questions to the given
passages and extract answer spans from them~\cite[inter
  alia]{seo2016bidirectional, yu2018qanet}.  Its popular solutions
include \abr{bidaf}, which matches the question and document passages
by bi-directional attention flows~\cite{seo2016bidirectional},
\abr{qan}{\small et}, which enriches the local contexts with global
self-attention~\cite{yu2018qanet}, and pre-training methods such as
\abr{bert}~\cite{devlin2018bert} and \abr{xln}{\small
  et}~\cite{yang2019xlnet}.

The most realistic models are those that also search for a passage:
%
\drqa{}~\cite{chen2017reading} retrieves documents from Wikipedia and
use \abr{mr} to predict the top span as the answer, and
\abr{orca}~\cite{lee-19} trains the retriever via an inverse cloze
task.
%
Nonetheless, questions mainly answerable by \drqa{} and \abr{orca} only require
single evidence from the candidate sets~\cite{min2018efficient}.
%
\name{} in contrast searches for edge evidence and nodes that can
answer the question; this subgraph often corresponds to the same
documents found by machine reading models.
%
Ideally, it would help synthesize information across \emph{multiple}
passages.


Multi-hop \abr{qa}, where answers require require assembling
information~\citep{welbl2018constructing, yang2018hotpotqa}, is a
\emph{task} to test whether machine reading systems can
synthesize information.
%
\abr{h}{\small otpot}\abr{qa}~\cite{yang2018hotpotqa} is the 
 multi-hop \abr{qa} benchmark: each answer is a text span requiring 
one or two hops.
%
Several models~\cite{qiu-etal-2019-dynamically,
  ding-etal-2019-cognitive, min2019multi} solve
this problem using multiple \abr{mr} models to extract multi-hop
evidence.
%
While we focus on datasets with Wikipedia entities as
answers, expanding \name{} to span-based answers (like \abr{h}{\small otpot}\abr{qa}) is a
natural future direction.



\subsection{Graph Networks for \abr{qa}}

\name{} is not the first to use graph neural networks~\cite[inter
  alia]{scarselli2009graph, kipf2016semi, schlichtkrull2017modeling}
for question answering.
%
\abr{e}{\small ntity}-\abr{gcn}~\cite{de2018question},
\abr{dfgn}~\cite{qiu-etal-2019-dynamically}, and
\abr{hde}~\cite{tu-etal-2019-multi} build the entity graph with entity
co-reference and co-occurrence in documents and apply \abr{gnn} to
the graph to rank the top entity as the answer.
%
\abr{c}{\small og}\abr{qa}~\cite{ding-etal-2019-cognitive} builds the
graph starting with entities from the question, then expanding the
graph using extracted spans from multiple \abr{mr} models as candidate
span nodes and adopt \abr{gnn} over the graph to predict the answer
from span nodes.
%
All these methods' edges are co-reference between entities or binary
scores about their co-occurrence in documents; \name{}'s primary
distinction is using free-text as graph edges which we then represent
and aggregate via a \abr{gnn}.

Other methods have learned representations of relationships between entities.
%
\abr{nubbi}~\cite{chang-09c} used an admixture over relationship
prototypes, while Iyyer et al.~\cite{iyyer-16} used neural dictionary
learning for analyzing literature.
%(including a temporal component for analyzing literature).
%
\name{} draws on these ideas to find similarities between 
passages and questions.


