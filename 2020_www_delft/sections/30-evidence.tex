
\section{Free-Text Graph Construction}
\label{sec:evidence}

%As discussed in previous sections, depending on current knowledge
Answering real world factoid questions with current knowledge
graphs falters when \kg{} relations lack coverage; we instead
use a free-text \kg{} to resolve the coverage
issue. One attempt to incorporate free-text corpus is to use Open
Information Extraction~\cite[\openie]{Lu:2019:ACQ} to extract
relations from natural language sentences as graph edges.
%
However \openie{} approaches
favor precision over recall, and heavily rely on well defined semantics, falling prey to the same
narrow scope that makes traditional \abr{kg} approaches powerful.
%within some domains but brittle outside of them.
%
Instead, we build the knowledge graph directly from free-text corpus,
represent indirect usings sentences which contain the two entities
(endpoints of the edge) as indirect relations.
%
We leave the \abr{qa} model (which only needs to output an answer
entity) to figure out \emph{which} sentences contain the information
to answer the question, eliminating the intermediate information
extraction step.

This section first discusses the \textit{construction} of a
\emph{free-text} knowledge graph and then \textit{grounding} questions
to it.

\subsection{Graph Construction}


The free-text knowledge graph uses the same nodes $V$ as
existing knowledge graphs: entities and their attributes, which can be
directly inherited from existing knowledge graphs.
%
The \tweennode{}s~$E$, instead of closed-form relations, are harvested natural
language sentences from a corpus.

\paragraph{Entity Nodes}

The graph inherits Wikipedia entities as the nodes $V$ (of course,
entities from other corpora could be used).
To represent the node, we use the first sentence of the corresponding
document as its \textit{node gloss}.

\paragraph{Free-Text Edges}

The \tweennode{}s between nodes are sentences that pairs of entities
in Wikipedia co-occur (again, other corpora such as
ClueWeb~\cite{callan2009clueweb09} could be used).
%
For specificity, let us find the edges that could link entities $a$ and $b$.
%
First, we need to know where entities appear.
%
Both $a$ and $b$ have their own Wikipedia pages---but that
does not give us enough information to find where the entities appear
\emph{together}.
%
TagMe~\cite{ferragina2010tagme} finds entities mentioned in free text;
we apply it to Wikipedia pages.
%
Given the entity linker's output, we collect the following sentences
as potential edges: (1) sentences in $a$'s Wikipedia page that mention
$b$, (2) sentences in $b$'s page that mention $a$, and (3) sentences
(anywhere) that mention both $a$ and $b$.

\subsection{Question Grounding}

Now that we have described the general components of our graph, we
next \textit{ground} a natural language question to a
subgraph of the \abr{kg}.  We then find an answer candidate in
this subgraph.

Specifically, for each question, we ground the full free-text
knowledge graph into a question-related subgraph
(Figure~\ref{fig:graph}): a bipartite graph with \emph{\leftnode{}s}
(e.g., \question{Dutch Golden Age}) on the left joined to
\emph{\rightnode{}s} on the right (e.g., \candidate{Delft}) via
\emph{\tweennode{}} (e.g., \edge{\candidate{Delft} played a highly
  influential role in the \question{Dutch Golden Age}}).

\paragraph{\leftnode{}s}

\name{} starts with identifying the entities in
question $\mathcal{X}_q$ as the \leftnode{}s $V_q=\{v_{i} \g v_i \in
\mathcal{X}_q\}$.
%
In Figure~\ref{fig:graph}, for instance, \question{Vermeer},
\question{The Little Street} and \question{Dutch Golden Age} appear in
the question; these \leftnode{}s populate the left side of
\name{}'s grounded graph. We use TagMe to identify question entities.

%the candidate still exists, but no edges to question entities, 
%so i slightly rephrase the next sentence.}

\paragraph{\rightnode{}s}

Next, our goal is to find \rightnode{}s.
In our free-text \abr{kg}, \rightnode{}s are the entities with 
 connections to the entities related to the question.
For example, \candidate{Delft} occurs in the Wikipedia page associated
with \leftnode{} \question{Vermeer} and thus becomes
a \rightnode{}.
The goal of this step is to build a set likely to
contain---has high coverage of---the answer entity.

We populate the \rightnode{} $V_a$ based 
on the following two approaches.
%
First, we \textit{link} entities contained in the Wikipedia \emph{pages} of
the \leftnode{}s to generate \rightnode{}s.
%
%Existing entity linking tools are 
%not perfect; they sometimes links to wrong entities.
%
To improve \rightnode{} recall, we next \textit{retrieve} entities:
after presenting the question text as a query to an \abr{ir} engine
(ElasticSearch~\cite{Gormley:2015:EDG:2904394}), the entities in the
top retrieved Wikipedia pages also become \rightnode{}s.

These two approaches reflect different ways entities are mentioned
in free text.
%
Direct mentions discovered by entity linking tools are straightforward
and often correspond to the clear \emph{semantic} information encoded in
Wikipedia (``In 1607, Khurram became engaged to Arjumand Banu Begum,
who is also known as Mumtaz Mahal'').
%
However, sometimes relationships are more \emph{thematic} and are not
mentioned directly; these are captured by \abr{ir} systems
(``She was a recognizable figure in academia, usually wearing a
distinctive cape and carrying a walking-stick'' describes
\underline{Margaret Mead} without named entities).
%
Together, these two approaches have good recall of the \rightnode{}
set $V_a$ (Table~\ref{tab:coverage}).

\paragraph{\tweennode{}s}

\name{} needs edges as evidence signals to know which \rightnode{} to select.
%
%This decision depends on the entities in the question and the text of
%the question.
%
The edges
connect \leftnode{}s $V_q$ to \rightnode{}s
$V_a$ with natural language.
%
In Figure~\ref{fig:graph}, the Wikipedia sentence
\edge{Vermeer was recognized during his lifetime in Delft} connects the
\leftnode{} \question{Vermeer} to the \rightnode{}
\candidate{Delft}.
%
%However, another \rightnode{} \candidate{Amsterdam}
%lacks an \tweennode{} from \leftnode{} \question{Vermeer} since no Wikipedia sentence
%connects them. 
Given two entities, we directly find the \tweennode{}s connecting them
from the free-text knowledge graph.


\paragraph{Final Graph}

Formally, for each question, our final graph of \name{} $G =(V, E)$
%$G =(V, E, \Phi(V), \Phi(E))$ includes
includes the follows:
%  
Nodes $V$ include \leftnode{}s $V_q$ and \rightnode{}s
$V_a$;
%
\tweennode{}s $E$ connect the nodes: each edge~$e$ contains Wikipedia
sentence(s) $\mathcal{S}({k}), k= 1,\dots, K$ linking the nodes.

\subsection{Question Graph Pruning} 


The graph grounding ensures high coverage of nodes and edges.
%
However, if the subgraph is \emph{too} big it slows training and inference.
% noise to nodes and edges.
\name{} prunes the graph with a simple filter to remove weak, 
spurious clues and to improve computational efficiency.

\paragraph{Candidate Entity Node Filter}

We treat the \rightnode{}s filtering as a ranking problem: given input
\rightnode{}s and question, score each connected \tweennode{} with a relevance score. 
%
During inference, the nodes with top-$K$ highest scores are kept (we choose the highest \tweennode{} score as the node relevance score), while
the rest are pruned. 
%
\name{} fine-tunes \abr{bert} as the filter model.\footnote{For efficiency, We use \abr{tfidf} filtering (top 1000 \tweennode{}s are kept) before \abr{bert}.}
%
To be specific, for each connected \tweennode{},
we concatenate the question and sentence, along with the \rightnode{}
gloss as the node context into \abr{bert}:
%\footnote{If the edge
%  sentences overflow the \bert{} maximum length, we truncate the
%  sentence field.}
%
%The inputs to the \abr{bert} are:
\begin{verbatim}
 [CLS] Question [SEP] Edge Sent [SEP] Node Gloss [SEP]
\end{verbatim}
We apply an affine layer and sigmoid 
activation on the last layer's 
[CLS] representation to produce scalar value.



During training, for each question, we use \tweennode{} connected to the answer node as positive
training example, and random sample $10$ negative \tweennode{}s as
negative examples. To keep the \name{}'s \abr{gnn} training efficient,
we keep the top twenty nodes 
 for training set. For more
comprehensive recall, development and test keep the top fifty nodes.
If the answer node is not kept, \name{} cannot answer the question.


\paragraph{Edge Sentence Filter}

Since there is no direct supervision signal for which edge sentences
are useful, we use \abr{tfidf}~\cite{salton1983introduction} to filter
sentences.
%
For all sentences $\mathcal{S}$ in \tweennode{} $e \in E$, we compute
each sentence's \abr{tfidf} cosine similarity to the question, and choose the
top five sentences for each \tweennode{}.
%We leave the details in Appendix.




