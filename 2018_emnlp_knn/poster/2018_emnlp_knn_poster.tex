\documentclass[final]{beamer} % beamer 3.10: do NOT use option hyperref={pdfpagelabels=false} !
%\documentclass[final,hyperref={pdfpagelabels=false}]{beamer} % beamer 3.07: get rid of beamer warnings
\mode<presentation> {  %% check http://www-i6.informatik.rwth-aachen.de/~dreuw/latexbeamerposter.php for examples
    \usetheme{I6dv}    %% you should define your own theme e.g. for big headlines using your own logos
    %\usetheme{Rochester}
}
\usepackage[english]{babel}
\usepackage[latin1]{inputenc}
\usepackage{epstopdf}
\usepackage{amsmath,amsthm, amssymb, latexsym}
\usepackage{soul}
\usepackage{bm}
\usepackage{caption}
%\usepackage{subcaption}
\usepackage{multirow}
\usepackage{extarrows}
\usepackage{xcolor}
\usepackage{colortbl}
\usepackage{tikz}
\usepackage{relsize}
\usepackage{hyperref}
\usepackage{booktabs}
\usepackage[normalem]{ulem}
\renewcommand{\ULthickness}{1.5pt}
\RequirePackage{natbib}

\newcommand{\indentitem}{\setlength\itemindent{35pt}}
\newcommand{\CC}[1]{\cellcolor{cyan!#1}}
\newcommand{\CCYellow}[1]{\cellcolor{yellow!#1}}
\newcommand{\highlight}[3]{\colorbox{#1!#2}{$\displaystyle#3$}}


\newcommand\wideDownarrow{\mathrel{\scalebox{4}[3]{$\Rightarrow$}}}

%\usepackage{times}
%\usefonttheme{structuresmallcapsserif}  % times is obsolete
\usefonttheme[onlymath]{serif}
%\boldmath
%\usepackage[orientation=landscape,size=custom,width=106.68,height=91.44,scale=1.4,debug]{beamerposter}
\usepackage[orientation=portrait,size=a0,scale=1.2,debug]{beamerposter}

\definecolor{olive20}{HTML}{E8E6BA}
\definecolor{red20}{HTML}{FBCBBB}

\setbeamertemplate{blocks}[rounded]

\newenvironment{variableblock}[3]{%
  \setbeamercolor{block body}{#2}
  \setbeamercolor{block title}{#3}
  \begin{block}{#1}}{\end{block}}

\usepackage{tikz}
\usepackage{xcolor}
\definecolor{color0}{HTML}{0000FF}
\definecolor{color1}{HTML}{3232FF}
\definecolor{color2}{HTML}{6666FF}
\definecolor{color3}{HTML}{9898FF}
\definecolor{color4}{HTML}{CCCCFF}
\definecolor{color5}{HTML}{FFFEFE}
\definecolor{color6}{HTML}{FFCCCC}
\definecolor{color7}{HTML}{FF9898}
\definecolor{color8}{HTML}{FF6666}
\definecolor{color9}{HTML}{FF3232}
\newcommand*{\mybox}[2]{\tikz[anchor=base,baseline=0pt,rounded corners=0pt, inner sep=0.2mm] \node[fill=#1] (X) {#2};}
\definecolor{clr0}{HTML}{FFFFFF}
\definecolor{clr1}{HTML}{D0E1F2}
\definecolor{clr2}{HTML}{B7D4EA}
\definecolor{clr3}{HTML}{94C4DF}
\definecolor{clr4}{HTML}{6AAED6}
\definecolor{clr5}{HTML}{4A98C9}
\definecolor{clr6}{HTML}{2E7EBC}
\definecolor{clr7}{HTML}{1764AB}
\definecolor{clr8}{HTML}{084A91}
\newcommand*{\mmybox}[2]{\tikz[anchor=base,baseline=0pt,rounded corners=0pt, inner sep=0.2mm] \node[fill=#1!60!white] (X) {#2};}

\newcommand{\abr}[1]{\textsc{#1}}
\newcommand{\dknn}{\textsc{DkNN}}
\newcommand{\lstm}{\textsc{LSTM}}
\newcommand{\bilstm}{\textsc{BiLSTM}}
\newcommand{\cnn}{\textsc{CNN}}
\newcommand{\loo}{leave-one-out}
\newcommand{\mb}[1]{\boldsymbol{\mathbf{#1}}}
\DeclareMathOperator*{\argmin}{\arg\!\min}
\DeclareMathOperator*{\argmax}{\arg\!\max}
\DeclareMathOperator*{\substr}{substr}
\DeclareMathOperator*{\subseq}{subseq}
\DeclareMathOperator*{\len}{len}
\definecolor{coloranswer}{HTML}{0596FF}
\definecolor{colorsquad}{rgb}{0,1,0}
\definecolor{colorsnli}{rgb}{1,0,0}
\definecolor{colorvqa}{rgb}{1,1,0}


\title[Fancy Posters]{\Huge Interpreting Neural Networks With Nearest Neighbors}
\author[]{\vspace{0.4cm} \huge Eric Wallace$^{\star}$, Shi Feng$^{\star}$, Jordan Boyd-Graber}
\institute[]
{
  \vspace{-1cm} \LARGE
  University of Maryland
  \vspace{1cm}
}

\begin{document}

\begin{frame}
\begin{columns}%[t]
\begin{column}{.5\linewidth}

% \begin{block}{Interpreting \abr{nlp} Classifiers}
% \center{Leave-one-out: remove a word and measure the confidence decrease~\citep{li2016visualizing}}
% \begin{equation*}
% \label{eq:importance}
%     I(w_i \mid \mb{x}, y) = f(y\mid\mb{x}) - f(y\mid\mb{x}_{-i}).
% \end{equation*}
% \end{block}

\begin{block}{Word Importance Using Leave-One-Out}
\vspace{0.3cm}

\begin{itemize}
\item Saliency maps indicate the ``importance'' of each word to the model's prediction.
\item Simple importance function: remove each word and measure the confidence decrease~\citep{li2016visualizing}.
\end{itemize}

\begin{figure}
\centering
\small
\begin{tabular}{p{0.6\columnwidth}cc}
Question & Confidence & Highlight \\\midrule
\mybox{clr0}{\strut{What}} \mybox{clr0}{\strut{did}}
\mybox{clr0}{\strut{Tesla}} \mybox{clr0}{\strut{spend}}
\mybox{clr0}{\strut{Astor's}} \mybox{clr0}{\strut{money}}
\mybox{clr0}{\strut{on}} \mybox{clr0}{\strut{?}} & \textbf{0.78} \\

\mybox{clr0}{\strut{\sout{What}}} \mybox{clr0}{\strut{did}}
\mybox{clr0}{\strut{Tesla}} \mybox{clr0}{\strut{spend}}
\mybox{clr0}{\strut{Astor's}} \mybox{clr0}{\strut{money}}
\mybox{clr0}{\strut{on}} \mybox{clr0}{\strut{?}} & 0.67 &
\mybox{clr2}{\strut{What}} \\

\mybox{clr0}{\strut{What}} \mybox{clr0}{\strut{\sout{did}}}
\mybox{clr0}{\strut{Tesla}} \mybox{clr0}{\strut{spend}}
\mybox{clr0}{\strut{Astor's}} \mybox{clr0}{\strut{money}}
\mybox{clr0}{\strut{on}} \mybox{clr0}{\strut{?}} & 0.72 &
\mybox{clr1}{\strut{did}} \\

\mybox{clr0}{\strut{What}} \mybox{clr0}{\strut{did}}
\mybox{clr0}{\strut{\sout{Tesla}}} \mybox{clr0}{\strut{spend}}
\mybox{clr0}{\strut{Astor's}} \mybox{clr0}{\strut{money}}
\mybox{clr0}{\strut{on}} \mybox{clr0}{\strut{?}} & 0.66 &
\mybox{clr2}{\strut{Tesla}} \\

\mybox{clr0}{\strut{What}} \mybox{clr0}{\strut{did}}
\mybox{clr0}{\strut{Tesla}} \mybox{clr0}{\strut{\sout{spend}}}
\mybox{clr0}{\strut{Astor's}} \mybox{clr0}{\strut{money}}
\mybox{clr0}{\strut{on}} \mybox{clr0}{\strut{?}} & 0.74 &
\mybox{clr1}{\strut{spend}} \\

\mybox{clr0}{\strut{What}} \mybox{clr0}{\strut{did}}
\mybox{clr0}{\strut{Tesla}} \mybox{clr0}{\strut{spend}}
\mybox{clr0}{\strut{\sout{Astor's}}} \mybox{clr0}{money}
\mybox{clr0}{\strut{on}} \mybox{clr0}{\strut{?}} & 0.76 &
\mybox{clr1}{\strut{Astor's}} \\

\mybox{clr0}{\strut{What}} \mybox{clr0}{\strut{did}}
\mybox{clr0}{\strut{Tesla}} \mybox{clr0}{\strut{spend}}
\mybox{clr0}{\strut{Astor's}} \mybox{clr0}{\strut{\sout{money}}}
\mybox{clr0}{\strut{on}} \mybox{clr0}{\strut{?}} & \textbf{0.48} &
\mybox{clr5}{\strut{money}} \\

\mybox{clr0}{\strut{What}} \mybox{clr0}{\strut{did}}
\mybox{clr0}{\strut{Tesla}} \mybox{clr0}{\strut{spend}}
\mybox{clr0}{\strut{Astor's}} \mybox{clr0}{\strut{money}}
\mybox{clr0}{\strut{\sout{on}}} \mybox{clr0}{\strut{?}} & 0.72 &
\mybox{clr1}{\strut{on}} \\

\mybox{clr0}{\strut{What}} \mybox{clr0}{\strut{did}}
\mybox{clr0}{\strut{Tesla}} \mybox{clr0}{\strut{spend}}
\mybox{clr0}{\strut{Astor's}} \mybox{clr0}{\strut{money}}
\mybox{clr0}{\strut{on}} \mybox{clr0}{\strut{\sout{?}}} & 0.73 &
\mybox{clr1}{\strut{?}} \\
\end{tabular}
\end{figure}
\vspace{0.2cm}
\begin{center}
\mybox{clr2}{\strut{What}} \mybox{clr1}{\strut{did}}
\mybox{clr2}{\strut{Tesla}} \mybox{clr1}{\strut{spend}}
\mybox{clr1}{\strut{Astor's}} \mybox{clr5}{\strut{money}}
\mybox{clr1}{\strut{on}} \mybox{clr1}{\strut{?}}
\end{center}
\vspace{0.3cm}
\end{block}

\begin{block}{Word Importance Using Gradient}
\vspace{0.3cm}
\begin{itemize}
\item Fast importance function: approximate a word's removal using the input gradient~\citep{simonyan2013deep}.
\end{itemize}
\begin{equation*}
    % \frac{\partial f}{\partial w_i} \
    % = \frac{\partial f}{\partial \bm{v}_i}\frac{\partial \bm{v}_i}{\partial w_i} \ 
    I(w_i \mid \mb{x}, y) = \frac{\partial f}{\partial \bm{v}_i} \cdot \bm{v}_i
\end{equation*}
\begin{itemize}
\item Computes importance of all words in one backward pass.
\end{itemize}
\vspace{0.3cm}
\end{block}

\begin{block}{Instability of Gradient-based Interpretation}
\vspace{0.3cm}

\begin{itemize}
\item Gradient interpretations are highly unstable to small input changes~\cite{feng2018rawr}.
\item At each step, remove the \emph{least} important word (\underline{underlined}).
\item ``advertisement'' goes most important to least important in one step. 
\end{itemize}

\vspace{0.3cm}
\center{\textbf{SQuAD}}
\vspace{0.3cm}
\begin{center}
% \tikz\node[fill=white!90!colorsquad,inner sep=1pt,rounded corners=0.3cm]{
\begin{tabular}{p{.8\columnwidth}}
Context: QuickBooks sponsored a ``Small Business Big Game'' contest, in
which Death Wish Coffee had a 30-second commercial aired free of
charge courtesy of QuickBooks. \mybox{coloranswer}{Death Wish
Coffee} beat out nine other contenders from across the United
States for the free advertisement.\\\\
Question:\\
 \mmybox{clr1}{\strut{What}} \mmybox{clr1}{\strut{company}} \mmybox{clr1}{\strut{won}} \mmybox{clr1}{\strut{free}} \mmybox{clr6}{\strut{advertisement}} \mmybox{clr1}{\strut{due}} \mmybox{clr1}{\strut{to}} \mmybox{clr1}{\strut{QuickBooks}} \mmybox{clr0}{\strut{\underline{contest}}} \mmybox{clr1}{\strut{?}}\\
 \mmybox{clr1}{\strut{What}} \mmybox{clr1}{\strut{company}} \mmybox{clr1}{\strut{won}} \mmybox{clr1}{\strut{free}} \mmybox{clr6}{\strut{advertisement}} \mmybox{clr1}{\strut{due}} \mmybox{clr1}{\strut{to}} \mmybox{clr0}{\strut{\underline{QuickBooks}}} \mmybox{clr1}{\strut{?}}\\
 \mmybox{clr2}{\strut{What}} \mmybox{clr0}{\strut{company}} \mmybox{clr1}{\strut{won}} \mmybox{clr1}{\strut{free}} \mmybox{clr0}{\strut{\underline{advertisement}}} \mmybox{clr1}{\strut{due}} \mmybox{clr1}{\strut{to}} \mmybox{clr5}{\strut{?}}\\
 \mmybox{clr1}{\strut{What}} \mmybox{clr0}{\strut{\underline{company}}} \mmybox{clr1}{\strut{won}} \mmybox{clr1}{\strut{free}} \mmybox{clr1}{\strut{due}} \mmybox{clr1}{\strut{to}} \mmybox{clr5}{\strut{?}}\\
 \mmybox{clr1}{\strut{What}} \mmybox{clr4}{\strut{won}} \mmybox{clr0}{\strut{\underline{free}}} \mmybox{clr1}{\strut{due}} \mmybox{clr0}{\strut{to}} \mmybox{clr5}{\strut{?}}\\
 \mmybox{clr1}{\strut{What}} \mmybox{clr1}{\strut{won}} \mmybox{clr1}{\strut{due}} \mmybox{clr1}{\strut{to}} \mmybox{clr0}{\strut{\underline{?}}}\\
\mmybox{clr1}{\strut{What}} \mmybox{clr4}{\strut{won}} \mmybox{clr1}{\strut{due}} \mmybox{clr0}{\strut{\underline{to}}}\\
\mmybox{clr4}{\strut{What}} \mmybox{clr2}{\strut{won}} \mmybox{clr1}{\strut{\underline{due}}}\\
 % \mmybox{clr2}{\strut{What}} \mmybox{clr1}{\strut{\underline{won}}}\\
 % \mmybox{clr2}{\strut{What}}\\

\end{tabular}
\end{center}
\vspace{0.3cm}
\end{block}

\begin{block}{Failure of Confidence-based Interpretation}
\vspace{0.3cm}

\begin{itemize}
\item Models have unreasonably high confidence on non-sensical inputs~\cite{feng2018rawr}
\item Confidence does not reflect model uncertainty $\to$ poor interpretation metric
\end{itemize}

\vspace{0.6cm}
\center{\textbf{VQA}} \\
\vspace{0.3cm}
% \tikz\node[fill=white!80!colorvqa,inner sep=1pt,rounded corners=0.3cm]{
\begin{tabular}{lp{0.5\columnwidth}p{0.17\columnwidth}}
& & \multirow{4}{*}{\includegraphics[height=6cm]{figures/vqa_example}}\\ 
Original & What color is the flower ? \\
Answer & yellow \\
\textbf{Reduced} & \textbf{flower ?} \\
Confidence & 0.827 $\to$ 0.819  \\
\end{tabular}
% }; % end of tikz

\center{\textbf{SNLI}}
\vspace{0.3cm}
\begin{center}
% \tikz\node[fill=white!90!colorsnli,inner sep=1pt,rounded corners=0.3cm]{
\begin{tabular}{lp{0.7\columnwidth}}
Premise & Well dressed man and woman dancing in the street \\
Original & Two man is dancing on the street \\
Answer & Contradiction \\
\textbf{Reduced} & \textbf{dancing} \\
Confidence & 0.977 $\to$ 0.706 \\
\end{tabular}
% }; % end of tikz

\end{center}
\vspace{0.3cm}
\end{block}

\begin{block}{Deep k-Nearest Neighbors (DkNN)}
\vspace{0.3cm}
%\begin{center}DkNN replaces the softmax classifier with k-nearest neighbors.\end{center}
\begin{itemize}
\item DkNN replaces the softmax classifier with k-nearest neighbors~\citep{papernot2018dknn}.
\item The \textit{conformity} score measures agreement among nearest neighbors.
\item Conformity better measures a model's uncertainty than softmax confidence.
\item \textbf{Our insight:} measure the change in conformity, not confidence, during leave-one-out.
\end{itemize}
\vspace{0.8cm}
\begin{center}
\includegraphics{figures/dknn-intuition.pdf}
\end{center}
\vspace{0.3cm}
\end{block}

\end{column} % end of left column
\begin{column}{.5\linewidth}

\begin{block}{DkNN for Text Classification}
\begin{itemize}
\item Train text classifiers (LSTM, CNN) using standard softmax output.
\item Replace softmax with kNN at each layer's final representation.
\item Does not affect text classification accuracy.
\end{itemize}
\begin{center}
\begin{tabular}{l|cccccc}
\toprule
& \abr{sst} & \abr{cr} & \abr{trec} & \abr{mpqa} & \abr{subj} \\
\midrule
\lstm{} & 86.7 & 82.7 & 91.5 & 88.9 & 94.8 \\
\lstm{} \dknn{} & 86.6 & 82.5 & 91.3 & 88.6 & 94.9 \\
\cnn{} & 85.7 & 83.3 & 92.8 & 89.1 & 93.5 \\
\cnn{} \dknn{} & 85.8 & 83.4 & 92.4 & 88.7 & 93.1 \\
\end{tabular}
\end{center}
\end{block}

\begin{block}{Interpreting Sentiment Analysis on SST}
\begin{itemize}
\item Comparing \textit{Conformity} Leave-one-out, \textit{Confidence} Leave-one-out, and input \textit{Gradient}.
\item Conformity leave-one-out ignores extraneous words such as ``clash'', ``fiction'', or ``Diane''.
\item Correctly identifies sentiment for ``terribly'' $\to$ trend holds across validation set.
\end{itemize}

\begin{table*}[t]
\centering
\begin{tabular}{lp{0.6\textwidth}}
\toprule
\textbf{Method} & \textbf{Saliency Map} \\
\midrule
Conformity &  \mybox{color5}{an} \mybox{color1}{intelligent} \mybox{color5}{fiction} \mybox{color5}{about} \mybox{color5}{learning} \mybox{color5}{through} \mybox{color5}{cultural} \mybox{color5}{clash}. \\
Confidence &  \mybox{color5}{an} \mybox{color1}{intelligent} \mybox{color6}{fiction} \mybox{color5}{about} \mybox{color5}{learning} \mybox{color5}{through} \mybox{color5}{cultural} \mybox{color9}{clash}. \\
Gradient &  \mybox{color5}{an} \mybox{color2}{intelligent} \mybox{color8}{fiction} \mybox{color5}{about} \mybox{color5}{learning} \mybox{color5}{through} \mybox{color5}{cultural} \mybox{color8}{clash}. \\
\midrule
Conformity &  \mybox{color5}{$<$Schweiger$>$} \mybox{color5}{is} \mybox{color2}{talented} \mybox{color5}{and} \mybox{color5}{terribly} \mybox{color2}{charismatic}. \\
Confidence &  \mybox{color6}{$<$Schweiger$>$} \mybox{color5}{is} \mybox{color2}{talented} \mybox{color5}{and} \mybox{color8}{terribly} \mybox{color2}{charismatic}. \\
Gradient &  \mybox{color6}{$<$Schweiger$>$} \mybox{color5}{is} \mybox{color2}{talented} \mybox{color5}{and} \mybox{color9}{terribly} \mybox{color2}{charismatic}. \\
\midrule
Conformity &  \mybox{color5}{Diane} \mybox{color5}{Lane} \mybox{color2}{shines} \mybox{color5}{in} \mybox{color5}{unfaithful}. \\
Confidence &  \mybox{color2}{Diane} \mybox{color5}{Lane} \mybox{color2}{shines} \mybox{color5}{in} \mybox{color6}{unfaithful}. \\
Gradient &  \mybox{color4}{Diane} \mybox{color5}{Lane} \mybox{color2}{shines} \mybox{color5}{in} \mybox{color8}{unfaithful}. \\
\bottomrule

\end{tabular}
\vspace{8pt}
\begin{tabular}{c}
Color Legend\quad  \mybox{color2}{Positive Impact}\quad \mybox{color7}{Negative Impact}
\end{tabular}
\caption{Comparison of interpretation approaches on \abr{sst} test
examples for the LSTM model. Blue indicates positive impact and red
indicates negative impact. Our method (\textit{Conformity}
\loo{}) has higher precision, rarely assigning
importance to extraneous words such as ``about'' or ``movie''.}
\label{table:saliency_maps}
\end{table*}

\end{block}
\begin{block}{Interpreting Textual Entailment on SNLI}
\begin{itemize}
\item SNLI contains dataset artifacts~\citep{gururangan2018annotation} that make hypothesis-only models strong.
\item Do normally trained models learn these artifacts?
\item Artifacts are assigned high importance (e.g., ``sleeping'' highly correlates with contradiction) 
\end{itemize}
\begin{center}
\begin{tabular}{llp{0.6\columnwidth}}
\toprule
\textbf{Prediction} & \textbf{Input} & \textbf{Saliency Map} \\
\midrule
\multirow{2}{*}{Contradiction} & Premise & a young boy reaches for and touches the propeller of a vintage aircraft.  \\
& Hypothesis &  \mybox{color5}{a} \mybox{color8}{young} \mybox{color6}{boy} \mybox{color2}{swims} \mybox{color4}{in} \mybox{color6}{his} \mybox{color4}{pool}. \\
\midrule
\multirow{2}{*}{Contradiction} & Premise & a brown dog and a black dog in the edge of the ocean with a wave under them boats are on the water in the background.  \\
& Hypothesis &  \mybox{color5}{the} \mybox{color5}{pets} \mybox{color5}{are} \mybox{color2}{sleeping} \mybox{color5}{on} \mybox{color5}{the} \mybox{color5}{grass}. \\
\midrule
& Premise & man in a blue shirt standing in front of a structure painted with geometric designs.\\
Entailment & Hypothesis &  \mybox{color5}{a} \mybox{color6}{man} \mybox{color5}{is} \mybox{color2}{wearing} \mybox{color5}{a} \mybox{color6}{blue} \mybox{color5}{shirt}.\\
\textcolor{red}{Entailment} & Hypothesis &  \mybox{color5}{a} \mybox{color6}{man} \mybox{color5}{is} \mybox{color2}{wearing} \mybox{color5}{a} \mybox{color6}{black} \mybox{color5}{shirt}.\\
\bottomrule

\end{tabular}
\vspace{8pt}
\begin{tabular}{c}
Color Legend\quad  \mybox{color2}{Positive Impact}\quad \mybox{color7}{Negative Impact}
\end{tabular}
% \caption{Interpretations generated with conformity \loo{} align with
% annotation biases identified in \abr{snli}. In the second example, the
% model puts emphasis on the word ``sleeping'', disregarding other words
% that could indicate the Neutral class. The final example diagnoses a
% model's incorrect Entailment prediction (shown in red). Green
% highlights indicate words that support the classification decision
% made (shown in parenthesis), pink highlights indicate words that
% support a different class.}
% \label{table:annotation_artifacts}
\end{center}

\end{block}

\begin{block}{Identifying SNLI Artifacts}
\begin{itemize}
\item We verify that known artifacts can be detected by our method.
\item Find the average importance rank of the artifacts for all appearances.
\item Conformity leave-one-out assigns higher average importance to dataset artifacts.
\end{itemize}
\begin{center}
% \setlength{\tabcolsep}{2pt}
\centering
\begin{tabular}{clcc}
\toprule
\textbf{Label} & \textbf{Artifact} & \textbf{Conformity} & \textbf{Confidence} \\
\midrule
\multirow{5}{*}{\textbf{Contradiction}}
& nobody & 1.00 & 1.00 \\
& sleeping & 1.64 & 2.34 \\
& no & 2.53 & 5.74 \\
& tv & 1.92 & 3.74 \\
& cat & 1.42 & 3.62 \\
\midrule
\multirow{5}{*}{\textbf{Neutral}}
& tall & 1.09 & 2.61 \\
& first & 2.14 & 2.99 \\
& competition & 2.33 & 5.56 \\
& sad & 1.39 & 1.79 \\
& favorite & 1.69 & 3.89 \\
\midrule
\multirow{5}{*}{\textbf{Entailment}}
& outdoors & 2.93 & 3.26 \\
& least & 2.22 & 4.41 \\
& instrument & 3.57 & 4.47 \\
& outside & 4.08 & 4.80 \\
& animal & 2.00 & 4.73 \\
\end{tabular}
% \caption{The top  artifacts identified by
% \citet{gururangan2018annotation} are shown on the left. For each word, we
% compute the average importance rank over the validation set using either
% \textit{Conformity} or \textit{Confidence} \loo{}. A score of 1 indicates that
% a word is always ranked as the most important word in the input.
% \textit{Conformity} \loo{} assigns higher importance to artifacts, suggesting
% it better diagnoses model biases.}
% \label{table:words}
\end{center}
\end{block}

\begin{block}{More please!}
\small{
\begin{itemize}
	\item{Paper}: \url{arxiv.org/abs/1809.02847}
    \item{Code}: \url{github.com/Eric-Wallace/deep-knn}
    \item{Blog}: \url{zerobatchsize.net}
    \item{More Interpretation Examples}: \url{sites.google.com/view/language-dknn}
\end{itemize}
}
\end{block}

\begin{block}
\scriptsize{
\bibliographystyle{plain}
\bibliography{../../bib/journal-abbrv,../../bib/fs,../../bib/jbg}
}
\end{block}
\end{column}
\end{columns}
\end{frame}
\end{document} 
