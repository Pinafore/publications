Humans predict future linguistic input before it is
observed~\cite{kutas-11}. This predictability has been formalized in
information theory~\cite{shannon-1948}---the more predictable a word
is, the lower the entropy---and has explained various linguistic
phenomena, such as garden path ambiguity~\cite{den-1997,hale-2001}.

Such instances of linguistic prediction are fundamental to statistical
\abr{nlp}. Auto-complete from search engines has made next-word
prediction one of best known \abr{nlp} applications.





Long-distance word prediction, such as verb prediction in \abr{sov}
languages~\cite{levy2013expectation,momma2015timing,chow2015bag}, is
important in simultaneous machine translation from
subject-object-verb~(\abr{sov}) languages to
subject-verb-object~(\abr{svo}) languages.  In \abr{svo} languages
such as English, for example, the main verb phrase usually comes after
the first noun phrase---the main subject---in a sentence, while in
verb-final languages such as Japanese or German, it comes very last.
Human simultaneous translators must make predictions about the
unspoken final verb to incrementally translate the sentence.
Minimizing interpretation delay thus requires making constant
predictions and deciding when to trust those predictions and commit to
translating in real-time.

Such prediction can also aid machines.  \newcite{matsubara-00} use
pattern-matching rules; \newcite{grissom2014} use a
statistical \ngram{} approach; and \newcite{oda2015acl} extend the
idea of using prediction by predicting entire syntactic constituents
for English-Japanese translation.  These systems require fast,
accurate verb prediction to further improve simultaneous translation
systems.  We focus on verb prediction in verb-final languages such as
Japanese with this motivation in mind.

In Section~\ref{sec:human}, we present what is, to our knowledge, the
first study of humans' ability to incrementally predict the verbs in
Japanese.  We use these human data as a yardstick to which to compare
computational incremental verb prediction.  Incorporating some of the
key insights from our human study into a discriminative
model---namely, the importance of case markers---
Section~\ref{sec:language_model} presents a better incremental verb
classifier than existing verb prediction schemes.  Having established
both human and computer performance on this challenging and
interesting task, Section~\ref{sec:related} reviews our work's
relationship to other studies in \abr{nlp} and linguistics.
