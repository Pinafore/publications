\section{Conclusions and Future Work}
\label{sec:conclusion}

We introduce \lexwsbmedrtm, a discriminative topic model that jointly
models topics and document links, detecting blocks in the document
network probabilistically by embedding the weighted stochastic block
model, rather via connected-components as in previous models.  A
separate Dirichlet prior for each block captures its topic
preferences, serving as an informed prior when inferring documents'
topic distributions. Max-margin learning learns to predict links from
documents' topic and word distributions and block assignments.

Our model better captures the connections and content of paper abstracts,
as measured by predictive link rank and topic quality.
\lexwsbmedrtm yields topics with better coherence, though not all
techniques contribute to the improvement. We support our
quantitative results with qualitative analysis looking at a pair of
example documents and at a pair of blocks, highlighting the robustness
of embedded \wsbm over blocks defined as \scc.

As next steps, we plan to explore model variations to support a wider range of
use cases. For example, although we have presented a version of the model
defined using undirected binary weight edges in the experiment, it would be
straightforward to adapt to model both directed/undirected and
binary/nonnegative real weight edges. We are also interested in modeling
changing topics and vocabularies~\cite{blei-06b,zhai-13}.  In the spirit of
treating links probabilistically, we plan to explore application of the model in
suggesting links that do not exist but should, for example in discovering missed
citations, marking social dynamics~\cite{nguyen-2014-sits-journal}, and identifying topically
related content in multilingual networks of
documents~\cite{Hu:Zhai:Eidelman:Boyd-Graber-2014}.
