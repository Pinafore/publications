% intro

\section{Orthogonal Cross-Lingual Mappings}\label{sec:intro}

Cross-lingual word embedding (\abr{clwe}) models map words from
multiple languages to a shared vector space, where words with similar
meanings are close, regardless of language.
\abr{clwe} is widely used in multilingual natural language
processing~\citep{klementiev-12,guo-15,zhang-16}.
Recent \abr{clwe} methods~\citep{ruder-17,glavas-19} independently train two
monolingual embeddings on large monolingual corpora and then align them
with a linear transformation.
Previous work argues that these transformations should be
\emph{orthogonal}~\citep{xing-15,smith-17,artetxe-16}: for any two words, the
dot product of their representations is the same as the dot product with the
transformation.
This preserves similarities and substructure of the original monolingual word
embedding but enriches the embeddings with multilingual connections between
languages.

\begin{figure*}[!t]
  \centering
  \includegraphics[width=.8\linewidth]{\figfile{girl.pdf}}
  \caption{The most similar Japanese words for \ja{少女} (sh\={o}jo ``girl'')
  and English words for ``girl'', measured by cosine similarity on Wikipedia
  fastText vectors, before (left) and after (right) \name{}.
  In the original embedding spaces, ``boy'' is the nearest neighbor for both
  languages but with a very different cosine similarity, and ``cat'' in English
  is not close to ``girl'': both violate the isomorphism assumed by an
  orthogonal transformation for cross-lingual representations.
  \name{} replaces \ja{猫} (neko ``cat'') with the more
  relevant \ja{美少女} (bish\={o}jo ``pretty girl'') and brings cosine similarities 
  closer.}
  \label{fig:example}
\end{figure*}

Thus, many state-of-the-art mapping-based \abr{clwe} methods impose an
orthogonal
constraint~\citep{artetxe-17,conneau-18,alvarez-18,artetxe-18b,ruder-18,alvarez-19}.
The success of orthogonal methods relies on the assumption that embedding spaces
are isomorphic; i.e., they have the same inner-product structures across
languages, but this does not hold for all
languages~\citep{sogaard-18,fujinuma-19}.
For example, English and Japanese fastText vectors~\citep{bojanowski-17} have
different substructures around ``girl'' (Figure~\ref{fig:example} left).
As a result, orthogonal mapping fails on some languages---when
\citet{hoshen-18} align fastText embeddings with orthogonal mappings, they
report 81\% English--Spanish word translation accuracy but only 2\% for the
more distant English--Japanese.

While recent work challenges the orthogonal
assumption~\citep{doval-18,joulin-18,jawanpuria-19}, we focus on whether simple
preprocessing techniques can \emph{improve the suitability of orthogonal
models}.
Our iterative method normalizes monolingual embeddings to make their structures
more similar (Figure~\ref{fig:example}), which improves subsequent
alignment.

Our method is motivated by two desired properties of monolingual embeddings
that support orthogonal alignment:
\begin{enumerate*}
\item Every word vector has the same length.
\item Each language's mean has the same length.
\end{enumerate*}
Standard preprocessing such as dimension-wise mean centering and length
normalization~\cite{artetxe-16} do not meet the two requirements at
the same time.
%
Our analysis leads to \emph{\name{}}, an alternating projection
algorithm that normalizes any word embedding to provably satisfy both
conditions.\footnote{The code is available at \url{https://github.com/zhangmozhi/iternorm}.}
%
After normalizing the monolingual embeddings, we then apply
mapping-based \abr{clwe} algorithms on the transformed embeddings.

We empirically validate our theory by combining \name{} with three
mapping-based \abr{clwe} methods.
%
\name{} improves word translation accuracy on a dictionary induction benchmark
across thirty-nine language pairs.
