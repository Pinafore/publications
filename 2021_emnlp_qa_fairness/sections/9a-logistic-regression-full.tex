\tablefile{logistic-regression-full-appendix}

\section{Logistic Regression features.}
\label{appendix:logistic-regression}
This section enlists a full set of features used for the logistic regression analysis after feature reduction, each with their coefficients, standard error, Wald Statistic and significance level in Table~\ref{tab:logistic-regression-appendix}. We also describe the templates and the implementation details of the features using in our logistic regression analysis (Section~\ref{subsec:logistic-regression}) in Appendix~\ref{appendix:feature-implementation}, and finally enlist some randomly sampled examples both from \nq{} and \triviaqa{} datasets in Appendix~\ref{appendix:multi-answers-examples} to show how \lrfeature{multi\_answers} feature has disparate effects on them.

\subsection{Implementation of Logistic Regression features}
\label{appendix:feature-implementation}
\begin{itemize}
  \item \lrfeature{q\_sim}: For closed-domain \qa{} tasks like \nq{} and \squad{}, this feature measures (sim)ilarity between (q)uestion text and evidence sentence---the sentence from the evidence passage which contains the answer text---using Jaccard similarity over unigram tokens~\cite{sugawara-18}. Since we do not include \squad{} in our logistic regression analysis (Section~\ref{subsec:logistic-regression}, this feature is only relevant for \nq{}.
  
  \item \lrfeature{e\_train\_count}: This binary feature represents if distinct (e)ntities appearing in a \qa{} example (through the approach described in Section~\ref{sec:mapping}) appears more than twice in the particular dataset's training fold. We avoid logarithm here as even the log frequency for some commonly occurring entities exceeds the expected feature value range.
  
  \item \lrfeature{t\_wh*}: This represents the features that captures the expected entity type of the answer: \lrfeature{t\_who}, \lrfeature{t\_what}, \lrfeature{t\_where}, \lrfeature{t\_when}. Each binary feature captures if the particular \lrfeature{"wh*"} word appears in the first ten (t)okens of the question text.\footnote{\qb{} questions often start with ``For 10 points, name this writer \textit{who}...''}
  
  \item \lrfeature{multi\_entities}: For number of linked person-entities in a example as described in Section~\ref{sec:mapping} as $n$, this feature is $log_2(n)$. Hence, this feature is 0 for example with just single person entity.
  
  \item \lrfeature{multi\_answers}: For number of gold-answers annotated in a example as $n$, this feature is $log_2(n)$. Hence, this feature is 0 for example with just answer.
  
  \item \lrfeature{g\_*}: Binary demographic feature signaling the presence of the (g)ender characterized by the feature. For instance, \lrfeature{g\_female} signals if the question is about a female person.
  
  \item \lrfeature{o\_*}: Binary demographic feature signaling the presence of the occupation (or profession) as characterized by the feature. For instance, \lrfeature{o\_writer} signals if the question is about a writer.
\end{itemize}