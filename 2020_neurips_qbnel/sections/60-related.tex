\section{Related Work}
\label{sec:rel}

Our work fits within the larger context of entity linking and question answering systems. 
In particular, we define a new version of entity linking that expand upon named entity linking. 
Despite the popularity of entity linking, there is little consensus amongst practitioners on how to precisely define Entity Linking~\cite{ling2015design}. 
This results in there being a variety of different ways to link entities, which depend on what the entities are used for~\cite{rosales2018should}.
Many versions of entity linking build upon named entity linking and develop a more difficult task, such as multilingual entity linking~\cite{raiman2018deeptype}. 
Other versions of entity linking include Wikification, which has Wikipedia as the external knowledge base~\cite{cheng2013relational}. 
Similarly, co-reference has been jointly accomplished with named entity recognition and entity linking~\cite{durrett2014joint}. 
Noun phrase linking is also related to implicit entity recognition, which has previously been studied in the context of references between tweets~\cite{hosseini2019implicit}. 
\\
\\
Entity linking has been used for a variety of applications, including question answering, such as the EARL system~\cite{dubey2018earl,dubey2016asknow}, which relies upon performing entity and relation linking at the same time. 
Additionally, entity linking is used along with knowledge graphs in order to answer questions~\cite{zhao2020delft}. 
Entity linking is also used for text understanding when used along with BERT~\cite{broscheit2020investigating}. 
Because entity linking is used for so many applications, developing a noun phrase linking dataset could potentially be useful for many downstream tasks. 

%Wikidia (based on hyperlinks)~\citep{ratinov2011wiki}, Wikia (based on hyperlinks)~\citep{logeswaran2019zero}, ACL2020~\citep{klie2020hero}