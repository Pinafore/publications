\section{Noun Phrase Linking}
\label{sec:gen}
We define the problem of Noun Phrase Linking and motivate the problem by discussing the improved performance in downstream tasks due to noun phrase linking. 
We additionally define guidelines for noun phrase linking and develop an interface for annotating documents using those guidelines in Section 3. 

\subsection{Noun Phrase Linking}
Named entities refer to specific nouns such as people's names, and the name of places. 
Annotating named entities provides a gain in accuracy when augmented to QA systems, but named entities exclude certain noun phrases that could further assist QA systems. 
In particular, resolving anaphoric references, such as resolving "One work by this author" to "Novum Organum", provides a more difficult task because of the lack of a direct link between the noun phrase and the entity to be linked. 
Resolving anaphoric references could allow for a bigger gain in helping QA systems, as seen by changing the answer to the correct answer, Francis Bacon, when noun phrases are replaced by their referenced entity (Figure 1).  
\\
\\
We define the task of noun phrase linking to be the union of annotating anaphoric references along with annotating named entities, and in effect, annotating all noun phrases in a document that link to a named entity.
This task differs from the coreference and entity linking, as entities that are referred to, but not necessarily ever mentioned in the document, can be linked. 
For example, within the first sentence of Figure 1, Novum Organum is never mentioned. 
However, "One work by this author" refers to Novum Organum, and so would be annotated in Noun Phrase Linking, but not in either coreference or entity linking.  
We develop guidelines for linking noun phrases (Section \ref{sec:int}), and plan to conduct experiments to determine the effect of annotating noun phrases upon QA performance (Section \ref{sec:exp}). 
\\
\\
We extend traditional entity linking to noun phrase linking because traditional entity linkers have been shown to perform well on \nel{} tasks, and extending to noun phrase linking allows for a more challenging task. 
Current models do well in NEL and coreference resolution, both of which are tasks related to noun phrase linking. 
Thus, it's plausible, although imperfect, that future models may show improvement on downstream tasks. 
We also propose experiments that explore the effect that noun phrase linking has upon question answering accuracy (Section \ref{sec:exp}), to determine the extent to which noun phrase linking assists question answering systems. 
 

\subsection{Noun Phrase Linking Guidelines}
We develop guidelines for determining which entities to link, and what to link them to. 
These guidelines will be used by annotators, along with examples, when determining what to annotate. 
We link text spans that are noun phrases and refer to a uniquely identifiable named entity in the knowledge base. 
For example, in Figure 1, we link "idols of the theatre" because it refers to a named entity. 
On the other hand, we don't link the word "symbols", despite the presence of a Wikipedia page for symbol, because smybol does not refer to a specific symbol, but rather the general word. 
If no Wikipedia page is present for a noun phrase, then we link it with "No Entity."
The "No Entity" links are subdivided into "No Entity Character" and "No Entity Literature" for characters and works of literature respectively.
Wikipedia is an incomplete knowledge base; rather than omit links simply because the correct entity does not exist, the entity should be linked, but assigned to a special null entity indicating its type. 



