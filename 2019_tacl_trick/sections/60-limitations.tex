\section{What Makes Adversarially-authored Questions Hard?}
\label{sec:limitations}

This section analyzes the
\challenge{}
questions to identify the source of their difficulty.

\subsection{Quantitative Differences in Questions}

One possible source of difficulty 
is data scarcity: the answers to adversarial questions rarely
appear in the training set. However, this is not the case;
the mean number of training examples per answer (e.g., \underline{George Washington})
is 14.9 for the adversarial questions versus
16.9 for the regular test data.

Another explanation for question difficulty
is limited 
``overlap'' with the training data, i.e.,
models cannot match $n$-grams
from the training clues. We measure the proportion of test $n$-grams
that also appear in training questions with the same answer
(Table~\ref{table:training_comparison}).
The overlap is roughly equal for unigrams but surprisingly higher for 
adversarial questions' bigrams.
The adversarial questions are also shorter
and have fewer \abr{ne}s.
However, the
proportion of named entities is
roughly equivalent. 

One difference between the questions written against
the \abr{ir} system and the ones written against the \abr{rnn} model
is the drop in \abr{ne}s. The decrease in \abr{ne}s is higher for \abr{ir} adversarial questions, which may explain their generalization: the \abr{rnn}
is more sensitive to changes in phrasing, while the \abr{ir} system is more
sensitive to specific words.

\setlength{\tabcolsep}{4pt}
\begin{table}[t]
\begin{tabular}{p{3.65cm}rr}
 \toprule

& \textbf{Adversarial} & \textbf{Regular}  \\
\midrule
Unigram overlap & 0.40 & 0.37 \\
Bigram overlap & 0.08 & 0.05 \\ 
Longest $n$-gram overlap & 6.73 & 6.87 \\
Average \abr{ne} overlap & 0.38 & 0.46 \\
\hspace{0.5cm} \abr{ir} Adversarial & 0.35 &  \\
\hspace{0.5cm} \abr{rnn} Adversarial & 0.44 &  \\
\midrule
Total Words & 107.1 & 133.5 \\
Total \abr{ne} & 9.1 & 12.5 \\
\bottomrule
\end{tabular}
\caption{The \challenge{} questions have similar $n$-gram overlap to
  the regular test questions. However, the overlap of the
  named entities (\abr{ne}) decreases
   for \abr{ir} Adversarial questions.}
\label{table:training_comparison}
\end{table}

\subsection{Categorizing Adversarial Phenomena}

\begin{table}[h]
\centering \small
\begin{tabular}{lrr}
 \toprule
Composing Seen Clues & 15\%\\
Logic \& Calculations & 5\% \\
Multi-Step Reasoning & 25\% \\
\midrule
Paraphrases & 38\% \\
Entity Type Distractors & 7\%\\
Novel Clues & 26\% \\
\midrule
Total Questions & \ntotalquestions{} \\
\bottomrule
\end{tabular}
\caption{A breakdown of the phenomena in the \challenge{} dataset.}
\label{table:stats}
\end{table}

We next qualitatively analyze \challenge{} questions. We
manually inspect
the author edit logs, classifying questions into six different
phenomena in
two broad categories (Table~\ref{table:stats})
from a random sample of 100 questions, double counting questions into multiple phenomena
when applicable.

\subsubsection{Adversarial Category 1: Reasoning}
\label{sec:compose_knowledge}


\begin{table*}[t]
\centering
\begin{tabular}{p{6.5cm}llp{2cm}}
\hline
         Question    & Prediction & Answer & Phenomenon                  \\ \hline

This man, who died at the Battle of the Thames, experienced a setback when his brother Tenskwatawa's influence over their tribe began to fade. & Battle of Tippecanoe & \underline{Tecumseh} & Composing Seen Clues \\ \hline

This number is one hundred fifty more than the number of Spartans at Thermopylae. & Battle of Thermopylae & \underline{450} & Logic \& Calculations \\ \hline

A building dedicated to this man was the site of the ``I Have A Dream'' speech. & Martin Luther King Jr. & \underline{Abraham Lincoln} & Multi-Step Reasoning \\ \hline

\end{tabular}
\caption{The first category of \challenge{} questions consists of examples that require reasoning. \emph{Answer} displays the correct answer (all models were incorrect). For these examples, connecting the training and \challenge{} clues is simple for humans but difficult for models.}
\label{table:unseen_sample}
\end{table*}

\begin{table*}[t]
\centering
\begin{tabular}{p{1.75cm}p{8.5cm}lp{2cm}}
\hline
         Set & Question    & Prediction      & Phenomenon                  \\ \hline

Training & Name this sociological phenomenon, the \emph{taking of one's own life}. & \underline{Suicide} & \multirow{2}{*}{Paraphrase} \\
Adversarial  & Name this \emph{self-inflicted method of death}. & \underline{Arthur Miller} & \\ \hline

Training &  Clinton played the \emph{saxophone on The Arsenio Hall Show}.  & \underline{Bill Clinton} &  \\ 
Adversarial & He was edited to appear in the film ``Contact''\dots\ For ten points, name this American president who played the \emph{saxophone on an appearance on the Arsenio Hall Show}. & \underline{Don Cheadle} & Entity Type Distractor \\ \hline

\end{tabular}
\caption{The second category of adversarial questions consists of clues that are present in the training data but are written in a distracting manner. \emph{Training} shows relevant snippets from the training data. \emph{Prediction} displays the \abr{rnn} model's answer prediction (always correct on Training, always incorrect on Adversarial).}
\label{table:rewrite_sample}
\end{table*}

The first question category requires reasoning about known clues (Table~\ref{table:unseen_sample}).~\smallskip

\paragraph{Composing Seen Clues:} These questions provide entities with a first-order
relationship to the correct answer. The system must 
triangulate the correct answer by ``filling in the blank''. For
example, the first question of Table~\ref{table:unseen_sample} names the
place of death of Tecumseh. The training data contains a
question his death reading
``though stiff fighting came from their Native American allies under
Tecumseh, who died at this battle'' (\underline{The Battle of the Thames}). The system must connect these two
clues to answer.~\smallskip

\paragraph{Logic \& Calculations:} These questions require mathematical
or logical operators. For example, the training data
contains a clue about the \underline{Battle of Thermopylae}: ``King
Leonidas and 300 Spartans died at the hands of the Persians''. The
second question in Table~\ref{table:unseen_sample} requires adding
150 to the number of Spartans.~\smallskip 

\paragraph{Multi-Step Reasoning:} This question type requires multiple reasoning
steps between entities. For example, the last question of
Table~\ref{table:unseen_sample} requires a reasoning step
from the ``I Have A Dream'' speech to the Lincoln Memorial and then 	another
reasoning step to reach \underline{Abraham Lincoln}. 

\subsubsection{Adversarial Category 2: Distracting Clues}
\label{sec:changes_language}

The second category consists of circumlocutory clues (Table~\ref{table:rewrite_sample}).~\smallskip

\paragraph{Paraphrases:} A common adversarial modification is to
paraphrase clues to remove exact $n$-gram matches from the training data. This
renders our \abr{ir} system useless but also hurts the neural models. Many of
the adversarial paraphrases go beyond syntax-only changes (e.g., the first row of Table~\ref{table:rewrite_sample}).~\smallskip

\paragraph{Entity Type Distractors:} Whether explicit or implicit in a model, one key component for \abr{qa} is determining the
answer type of the question. Authors take advantage of this
by providing clues that cause the model to select the wrong answer type. For example,
in the second question of Table~\ref{table:rewrite_sample}, the ``lead-in'' clue implies
the answer may be an actor. The \abr{rnn} model answers Don Cheadle in response despite previously
seeing the Bill Clinton ``playing a saxophone'' clue in the training data.~\smallskip

\paragraph{Novel Clues:} Some \challenge{} questions are hard
not because of phrasing or logic but because our models have
not seen these clues.  These questions are easy
to create: users can add \emph{Novel Clues} that---because they are
not uniquely associated with an answer---confuse the models.
While not as linguistically interesting, novel clues are
not captured by Wikipedia or \qb{} data, thus improving the
dataset's diversity.  For example, adding clues about literary
criticism~\cite{hardwick-67,watson-96} to a question about Lillian
Hellman's \underline{The Little Foxes}: ``Ritchie Watson commended
this play's historical accuracy for getting the price for a dozen eggs
right---ten cents---to defend against Elizabeth Hardwick's contention
that it was a sentimental history.'' Novel clues create 
an incentive for models to use
information beyond past questions and Wikipedia.

Novel clues have different effects on \abr{ir} and neural models:
while \abr{ir} models largely ignore them, novel clues can lead
neural models astray.  For example, on a question about \underline{Tiananmen
Square}, the \abr{rnn} model buzzes on
the clue ``World Economic Herald''.  However, adding a novel clue
about ``the history of shaving''
renders the brittle \abr{rnn} unable to buzz on the ``World Economic
Herald'' clue that it was able to recognize before.\footnote{The ``history of shaving'' is a tongue-in-cheek name for a poster
displaying the hirsute leaders of Communist thought. It goes from the bearded Marx and Engels,
to the mustachioed Lenin and Stalin, and finally the clean-shaven Mao.}  This helps
to explain why \challenge{} questions written against the \abr{rnn}
do not stump \abr{ir} models.