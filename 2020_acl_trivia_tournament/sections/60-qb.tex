

\section{Why \qb{} is the Gold Standard}
\label{sec:qb}

We now focus our thus far wide-ranging \abr{qa} discussion to a specific format: \qb{}, which has many of the desirable properties outlined above.
We have no delusion that mainstream \abr{qa} will universally adopt this format (indeed, a monoculture would be bad).
However, given the community's emphasis on fair evaluation, computer \abr{qa} can borrow \emph{aspects} from the gold standard of human \abr{qa}.

We have shown example of \qb{} questions, but we have not explained how the format works; see \newcite{DBLP:journals/corr/abs-1904-04792} for more.
You might be scared off by how long the questions are.
However, in real \qb{} trivia tournaments, they are not finished because the questions are \emph{interruptible}.

\paragraph{Interruptible}

A moderator reads a question.
Once someone knows the answer, they use a signaling device to ``\emph{buzz in}''.
If the player who buzzed is right, they get points.
Otherwise, they lose points and the question continues for the other team.  

Not all trivia games with buzzers have this property, however.
For example, take \jeopardy{}, the subject of Watson's \textit{tour de force}~\cite{ferruci-10}.  
While \jeopardy{} also uses signaling devices, these only work \emph{once the question has been read in its entirety}; Ken Jennings, one of the top \jeopardy{} players (and also a \qb{}er) explains it on a \textit{Planet Money} interview~\cite{malone-19}: \clearpage
\begin{quote}
{\bf Jennings:} The buzzer is
    not live until Alex finishes reading the question. And if you buzz
    in before your buzzer goes live, \emph{you actually lock yourself out
    for a fraction of a second}. So the big mistake on the show is
    people who are all adrenalized and are buzzing too quickly, too
    eagerly. \\
{\bf Malone:} \abr{ok}. To some degree, \jeopardy{} is kind of a video game, and a \emph{crappy video game where it's, like, light goes on, press button}---that's it. \\
{\bf Jennings:} (Laughter) Yeah. \\
\end{quote}
\jeopardy{}'s buzzers are a gimmick to ensure good television; however, \qb{} buzzers discriminate knowledge (Section~\ref{sec:discriminative}).
Similarly, while \triviaqa{}~\cite{joshi-17} is written by knowledgeable writers, the questions are not pyramidal.

\paragraph{Pyramidal}
\label{sec:pyramidality}

Recall that effective datasets discriminate the best
from the rest---the higher the proportion of effective
questions~$\rho$, the better.
\qb{}'s~$\rho$ is nearly 1.0 because discrimination happens
\emph{within} a question: after every word, an answerer must decide
if they know enough to answer.
\qb{} questions are arranged so that questions are maximally \emph{pyramidal}: questions begin with hard clues---ones that require deep understanding---to more accessible clues that are well known.


\paragraph{Well-Edited}

\qb{} questions are created in phases.
First, the \emph{author} selects the answer and assembles (pyramidal) clues.
A \emph{subject editor} then removes ambiguity, adjusts acceptable answers, and tweaks clues to optimize discrimination.
Finally, a \emph{packetizer} ensures the overall set is diverse, has uniform difficulty, and is without repeats.

\paragraph{Unnatural}
\label{sec:unnatural}

Trivia questions are fake: the asker already knows the answer.  
But they're no more fake than a course's final exam, which---like leaderboards---are designed to test knowledge.

Experts know when questions are ambiguous (Section~\ref{sec:ambiguity}); while ``what play has a character whose father is dead'' could be \textit{Hamlet}, \textit{Antigone}, or \textit{Proof}, a good writer's knowledge avoids the ambiguity.
When authors omit these cues, the question is derided as a ``hose''~\cite{2013-eltinge}, which robs the tournament of fun (Section~\ref{sec:fun}).

One of the benefits of contrived formats is a focus on specific phenomena. 
\newcite{dua-19} exclude questions an existing \abr{mrqa} system could answer to focus on challenging quantitative reasoning.
One of the trivia experts consulted in \newcite{wallace-19} crafted a question that tripped up neural \abr{qa} by embedding the phrase ``this author opens Crime and Punishment'' into a question; the top system confidently answers \underline{Fyodor Dostoyevski}.
However, that phrase was in a longer question ``The narrator in \textit{Cogwheels} by this author opens \textit{Crime and Punishment} to find it has become \textit{The Brothers Karamazov}''. 
Again, this shows the inventiveness and linguistic dexterity of the trivia community.

A counterargument is that real-life questions---e.g., on Yahoo!
Questions~\cite{szpektor-13}, Quora~\cite{iyer-17} or web
search~\cite{kwiatkowski-19}---ignore the craft of question writing.
Real humans react to unclear questions with confusion or divergent
answers, explicitly answering with how they interpreted the original
question (``I assume you meant\dots'').

Given real world applications will have to deal with the inherent noise and ambiguity of unclear questions, our systems must cope with it. 
However, addressing the real world cannot happen by glossing over its complexity.




\paragraph{Complicated} 

\qb{} is more complex than other datasets.  
Unlike other datasets where you just need to decide \emph{what} to
answer, in \qb{} you also need to choose \emph{when} to answer the
question.\footnote{This complex methodology can be an advantage.  The
  underlying mechanisms of systems that can play \qb{} (e.g.,
  reinforcement learning) share properties with other tasks, such as
  simultaneous
  translation~\cite{grissom:he:boyd-graber:morgan-2014,ma-etal-2019-stacl},
  human incremental processing~\cite{levy-08,levy-11}, and opponent
  modeling~\cite{he-16}.}
While this improves the dataset's discrimination, it can hurt
popularity because you cannot copy/paste code from other \abr{qa}
tasks.
The cumbersome pyramidal structure complicates\footnote{But does not
  necessarily preclude, as the Illinois High School
  Scholastic Bowl Coaches Association shows:
\begin{quote}
    This is the smallest counting number which is the radius of a sphere whose volume is an integer multiple of $\pi$. It is also the number of distinct real solutions to the equation $x^7-19x^5=0$. This number also gives the ratio between the volumes of a cylinder and a cone with the same heights and radii. Give this number equal to the log base four of sixty-four.
\end{quote}} some questions (e.g.,
what is log base four of sixty-four).



