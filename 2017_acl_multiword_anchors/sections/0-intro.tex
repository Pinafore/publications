\section*{}
\label{sec:intro}

Topic models distill large collections of text into topics,
giving a high-level summary of the thematic structure of the data without
manual annotation.
In addition to facilitating discovery of topical
trends~\cite{topical-guide}, topic modeling is used for a wide
variety of problems including
document classification~\cite{topic-classification},
information retrieval~\cite{lda-ir},
author identification~\cite{author-topic},
and sentiment analysis~\cite{topic-sentiment}.
However, the most compelling use of topic models is to help users understand
large datasets~\cite{termite}.

Interactive topic modeling~\cite{hu-14:itm} allows non-experts to refine
automatically generated topics, making topic models less of a ``take it or
leave it'' proposition.
Including humans input during training improves the quality of the
model and allows users to guide topics in a specific way, custom
tailoring the model for a specific downstream task or
analysis.

The downside is that interactive topic modeling is slow---algorithms typically
scale with the size of the corpus---and requires non-intuitive information from
the user in the form of must-link and cannot-link
constraints~\cite{andrzejewski-09}.
We address these shortcomings of interactive topic modeling by using an
interactive version of the \emph{anchor words} algorithm for topic models.

The anchor algorithm~\cite{anchors-practical} is an alternative topic modeling
algorithm which scales with the number of
unique word types in the data rather than the number of documents or tokens
(Section~\ref{sec:vanilla-algo}).
This makes the anchor algorithm fast enough for interactive use, even in
web-scale document collections.

A drawback of the anchor method is that anchor words---words that have high
probability of being in a \emph{single} topic---are not intuitive.
We extend the anchor algorithm to use multiple anchor words in tandem (Section
\ref{sec:multiword-extension}).
Tandem anchors not only improve interactive refinement, but also make the
underlying anchor-based method more intuitive.

For interactive topic modeling,
tandem anchors produce higher quality topics than single word anchors
(Section~\ref{sec:oraclular-experiments}).
Tandem anchors provide a framework for fast interactive topic modeling: users
improve and refine an existing model through multiword anchors
(Section~\ref{sec:interactive-experiments}).
Compared to existing methods such as Interactive Topic Models~\cite{hu-14:itm},
our method is much faster.
