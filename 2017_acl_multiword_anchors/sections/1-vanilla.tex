\section{Vanilla Anchor Algorithm}
\label{sec:vanilla-algo}

The anchor algorithm computes the topic matrix $\bm{A}$,
where $\bm{A}_{v,k}$ is the conditional probability of observing word $v$ given
topic $k$, e.g., the probability of seeing the word ``lens'' given the
\underline{camera} topic in a corpus of Amazon product reviews.
\newcite{nmf-provably} find these probabilities by assuming that every topic
contains at least one `anchor' word which has a non-zero probability only in
that topic.
Anchor words make computing the topic matrix $\bm{A}$ tractable because the
occurrence pattern of the anchor word mirrors the occurrence pattern of the
topic itself.

To recover the topic matrix $\bm{A}$ using anchor words, we first compute a $V
\times V$ cooccurrence matrix $\bm{Q}$, where $\bm{Q}_{i,j}$ is the conditional
probability $p(w_j \g w_i)$ of seeing word type $w_j$ after having seen $w_i$ in the
same document.
A form of the Gram-Schmidt process on $\bm{Q}$ finds anchor words $\{g_1 \dots
g_k\}$~\cite{anchors-practical}.

Once we have the set of anchor words, we can compute the probability of a topic
given a word (the inverse of the conditioning in $\bm{A}$).
This coefficient matrix $\bm{C}$ is defined row-wise for each word~$i$
\begin{equation}
\bm{C}^*_{i,\cdot} = \argmin_{C_{i,\cdot}} D_{KL} \left( \bm{Q}_{i,\cdot} \bigg\| \sum_{k=1}^K \bm{C}_{i,k} \bm{Q}_{g_k,\cdot} \right),
\label{eq:solve-c}
\end{equation}
which gives the best reconstruction (based on \mbox{Kullback-–Leibler}
divergence $D_{KL}$) of non-anchor words given anchor words' conditional probabilities.
For example, in our product review data, a word such as ``battery'' is a convex combination of the anchor words' contexts ($
\bm{Q}_{g_k,\cdot}$) such as ``camera'', ``phone'', and ``car''.
Solving each row of $C$ is fast and is embarrassingly parallel.
Finally, we apply Bayes' rule to recover the topic matrix $\bm{A}$ from the
coefficient matrix $\bm{C}$.

The anchor algorithm can be orders of magnitude faster
than probabilistic inference~\cite{anchors-practical}.
The construction of $\bm{Q}$ has a runtime of $O(D N^2)$ where $D$ is the
number of documents and $N$ is the average number of tokens per document.
This computation requires only a single pass over the data and can
be pre-computed for interactive use-cases.
Once $\bm{Q}$ is constructed, topic recovery requires $O(K V^2 + K^2 V I)$,
where $K$ is the number of topics, $V$ is the vocabulary size, and $I$ is the
average number of iterations (typically 100--1000).
In contrast, traditional topic model inference typically
requires multiple passes over the entire data.
Techniques such as Online \textsc{lda}~\cite{online-lda} or Stochastic Variation
Inference~\cite{stochastic-variational} improves this to a single
pass over the entire data.
However, from Heaps' law~\cite{heaps-law} it follows that $V^2 \ll D N$ for large datasets,
leading to much faster inference times for anchor methods compared to
probabilistic topic modeling.
Further, even if online inference could incorporate human guidance, a
single pass takes too long for interactive use.
