\section{Defining Question-In-Context Rewrites  \label{sec:pp_task}}



We formally define the task of question-in-context rewriting
(de-contextualization).
Given a conversation topic $t$ and a history $H$ of $m-1$ turns, 
each turn $k$ is a question $q_i$ and an answer $a_i$;
the task is to generate a rewrite $q'_m$ for the next question $q_m$
based on $H$.
Since $q_m$ is part of the conversation, its meaning often involves
references to parts of its preceding history.
A valid rewrite $q'_m$ should be self-contained: a correct answer
to $q'_m$ by itself is a correct answer to $q_m$ combined with the
question's preceding history $H$.

Figure~\ref{fig:example_convo} shows the assumptions of \abr{cqa} and
how they are made explicit in rewrites.
The first question omits the title of the page (Anna Vissi), the
second question omits the answer to the first question (replacing both
Anna Vissi and her husband with the pronoun ``they''), and the last
question adds a scalar implicature that must be resolved.



