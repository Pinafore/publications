\begin{frame}{\nlp{} for Low-resource Languages}
\begin{itemize}
\item Scarcity of both labeled and unlabeled data holds back applications in
    low-resource languages
\item Cross-lingual word embeddings (\abr{clwe}) can bridge the gap by mapping
    words from different languages to a shared vector space
\end{itemize}
\pause
\textbf{How can we quickly refine CLWE for low-resource NLP?}
\end{frame}


\begin{frame}{Refining \abr{clwe}}
\begin{center}
\begin{figure}
\begin{overprint}
    \onslide<1>\centerline{\includegraphics[width=0.8\textwidth]{\figfile{map_fr_bad.pdf}}}
    \onslide<2>\centerline{\includegraphics[width=0.8\textwidth]{\figfile{map_fr_good.pdf}}}
    \onslide<3->\centerline{\includegraphics[width=0.8\textwidth]{\figfile{map_fr.pdf}}}
\end{overprint}
\end{figure}
\onslide<3>
    \textbf{Classification clime:} Areas in embedding space where words induce similar
    labels for a task
\end{center}
\end{frame}


\begin{frame}{\name{}}
    CLassifying Interactively with Multilingual Embeddings
    \begin{enumerate}
        \item<2-> Select keywords with \textit{gradient-based
            salience}~\citep{li-16}
        \item<3-> Collect user feedback
        \item<4-> Refine embeddings on user feedback through
            \textit{retrofitting}~\citep{mrksic-17}
    \end{enumerate}
\end{frame}

