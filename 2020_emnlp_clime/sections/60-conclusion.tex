\section{Conclusion and Future Work}

\name{} is an interactive system that enhances \abr{clwe} for a
 task by asking a bilingual speaker for word-level similarity
annotations.
We test \name{} on cross-lingual information triage in international health emergencies for four
low-resource languages.
Bilingual users can quickly improve a model with the help of \name{} at a
faster rate than an active learning baseline.
Combining active learning with \name{} further improves the system.

\name{} has a modular design with three components: keyword ranking, user
interface, and embedding refinement.
The keyword ranking and the embedding refinement modules build upon existing
methods for interpreting neural networks~\citep{li-16} and fine-tuning word
embeddings~\citep{mrksic-17}.
Therefore, future advances in these areas may also improve \name{}.
Another line of future work is to investigate alternative user interfaces.
For example, we could ask bilingual users to \emph{rank} nearest
neighbors~\citep{sakaguchi-18} or provide scalar grades~\citep{hill-15} instead
of accepting/rejecting individual neighbors.

We also explore a simple combination of active learning and
\name{}. Simultaneously applying both methods is better than using either alone.
In the future, we plan to train a policy that dynamically combines the two interactions with reinforcement learning~\citep{fang-17}.
