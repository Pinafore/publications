% active learning
\begin{frame}{Method: Active Learning}
    \begin{itemize}
        \item<1-> Use active learning to find particular spans of text for users to label
    %\centerline{\includegraphics[width=\textwidth]{\figfile{example_target.pdf}}}
        \item<2-> The goal is to adapt the model to the target domain by continue
training it on spans labeled from active learning
    \end{itemize}
\end{frame}


\begin{frame}{What should we label?}
    \onslide<1->
       \begin{quote}
           A fantastic \spot<2>[fill=green]{experience}, very informative, very
           \spot<2>[fill=green]{time} consuming but enjoyable. So much information to take
           in about Guinness that \spot<4>[fill=red]{you} would’ve never known. For
           example, \spot<3>[fill=Orchid]{the brewery} hired the statistician Willam Gosset
           in 1899. \spot<3>[fill=Orchid]{The “student”} was known for developing
           the Student’s t-test,
           \spot<5>[fill=SkyBlue]{a well-known method in statistical
           inference}.
       \end{quote}
       \centering
       \medskip
    \only<2>{\textcolor{green}{Uncertainty in mention detection}}
    \only<3>{\textcolor{Orchid}{Uncertainty in mention clustering}}
    \only<4>{\textcolor{red}{Uncertainty in mention clustering
    conditioned on mention detection}}
    \only<5>{\textcolor{SkyBlue}{Uncertainty in both mention detection
    and mention clustering}}
\end{frame}

%\begin{frame}{Method: Active Learning}
%We explore active learning for adapting CR models by:
%\begin{enumerate}
%\onslide<1-> \item Sampling spans according to different sources of model
%uncertainty
%\onslide<2-> \item Understanding the trade-off between reading and labeling
%costs
%\end{enumerate}
%\end{frame}

