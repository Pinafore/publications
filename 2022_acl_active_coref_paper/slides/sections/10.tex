% introduce coreference
\begin{frame}{Coreference Resolution (CR)}
\onslide<1-> The task of discovering spans of text that refer to the same entity
\vfill
    \centerline{\includegraphics[width=\textwidth,
    frame]{\figfile{example_src.pdf}}}
\vfill
\onslide<2-> Neural, end-to-end models~\citep{lee-2018, joshi-2020} are
    SOTA for OntoNotes 5.0 \\

\end{frame}

% introduce problem of adapting coreference
\begin{frame}{Problem: Adapting CR Models}
\onslide<1-> Models trained on OntoNotes may not immediately
    generalize to new domains
    \vfill
    \centerline{\includegraphics[width=\textwidth,frame]{\figfile{example_src.pdf}}}
    \vfill
    \centerline{\includegraphics[width=\textwidth,frame]{\figfile{example_tgt.pdf}}}
    \vfill
\onslide<2-> Impedes immediate application for scenarios like distinguishing
    entities in scientific articles
\end{frame}

\begin{frame}{Problem: Adapting CR Models}
    \begin{itemize}
        \onslide<1-> \item \citep{xia-2021} show the benefits of \emph{continued training}
    where a model trained on OntoNotes is further trained on the target dataset
            \vfill
    \onslide<2-> \item However, they assume labeled data already exist in the target
        domain
            \vfill
    \onslide<3-> \item How can we adapt CR models without requiring large amounts of
        newly annotated data?
    \end{itemize}
\end{frame}


