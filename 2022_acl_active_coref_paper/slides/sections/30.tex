% generalizing entropy
%\begin{frame}{How to quantify uncertainty?}
%\begin{itemize}
%\onslide<1-> \item A common uncertainty sampling strategy is maximum entropy
    %sampling~\citep{lewis-1994}, which chooses data with
%highest predictive entropy
%\onslide<2-> \item For coreference resolution, these strategies are not as straightforward because
%the model has different scores for mention detection and clustering
%\end{itemize}
%\end{frame}


%\begin{frame}{Decomposing Decisions in CR}

%\begin{enumerate}
%\onslide<1-> \item \textbf{Mention detection:}
    %\begin{itemize}
        %\item Given a span of text $x$, let $X$ be the random variable encoding whether $x$ is an
%entity mention (1) or not (0)
        %\item The CR model first scores $P(X=1)$, which is the likelihood of $x$
            %being an entity mention
    %\end{itemize}
%\onslide<2-> \item \textbf{Mention clustering:}
    %\begin{itemize}
        %\item Given span $x$, let $C$ be the random variable associated with the entity
%cluster of $x$
        %\item For spans that are likely entity mentions, the CR
%model determines $P(C=c \given X=1)$ for each observed entity cluster $c$
    %\end{itemize}
%\end{enumerate}
%\end{frame}

%\begin{frame}{Uncertainty Sampling Strategies}
%\textbf{ment-ent:} Mention detection entropy
%\begin{align}
    %\ent{MENT}(x)
        %&= \ent{}{(X)} \\
        %&= - \sum_{i=0}^1 P(X=i) \log P(X=i). \nonumber
%\end{align}
%Samples spans that challenge mention detection (e.g. class-ambiguous words like
%``park'').
%\end{frame}

%\begin{frame}{Uncertainty Sampling Strategies}
%\textbf{clust-ent:} Mention clustering entropy
%\begin{align}
    %\ent{CLUST}(x)
        %&= \ent{}(C \given X = 1) \\
        %&= -\sum_{c \in \mathcal{C}} P(C = c \given X =
    %1) \log \nonumber \\
    %& \qquad P(C = c \given X = 1). \nonumber
%\end{align}
%Entropy computation does not explicitly address uncertainty in mention
%detection.
%\end{frame}

%\begin{frame}{Uncertainty Sampling Strategies}
%\textbf{cond-ent:} Conditional entropy
%\begin{equation}
%\begin{split}
    %\ent{COND}(x)
        %&= \ent{}(C \given X) \\
        %&= \sum_{i=0}^1 P(X=i) \ent{}(C \given X = i) \\
        %&= P(X=1) \ent{}(C \given X=1) \\
        %&= P(X=1) \ent{CLUST}(x).
%\end{split}
%\end{equation}
%Samples words like pronouns because they are obviously entity mentions but
%difficult to cluster.
%\end{frame}


%\begin{frame}{Uncertainty Sampling Strategies}
%\textbf{joint-ent:} Joint entropy
%\begin{align}
    %\ent{JOINT}(x)
        %&= \ent{}(X,C) = \ent{}(X) + \ent{}(C \given X) \nonumber \\
        %&= \ent{MENT}(x) + \ent{COND}(x).
%\end{align}
%Samples spans that are difficult to detect as entity mentions \emph{and}
%too confusing to cluster.
%\end{frame}

\begin{frame}{Experiments: Strategies}
\onslide<1->
\begin{enumerate}
\item \textbf{ment-ent:} Mention detection entropy
\item \textbf{clust-ent:} Mention clustering entropy
\item \textbf{cond-ent:} Conditional entropy
\item \textbf{joint-ent} Joint entropy
\onslide<2->
\item \textbf{random:} Randomly sample from all spans in the document
\onslide<3->
\item \textbf{random-ment:} Randomly sample only from
    the pool of spans that are likely entity mentions
\end{enumerate}
\end{frame}

\begin{frame}{Experiments: Datasets}
    \begin{enumerate}
        \item \textbf{OntoNotes 5.0 (source):}
            \begin{itemize}
                \item Most common CR dataset~\citep{pradhan-2013}
            \item News articles and telephone
            conversations
            \item Only non-singletons are
            annotated
            \end{itemize}
            \vspace{1cm}
        \item \textbf{PreCo (target):}
            \begin{itemize}
            \item Large corpus of grade-school reading comprehension
        texts~\citep{chen-2018}
            \item Singletons are annotated
            \end{itemize}
    \end{enumerate}
\end{frame}

% comparing strategies
\begin{frame}{Comparing Strategies}
\begin{columns}[c]
\column{0.6\textwidth}
\only<1>{
    \includegraphics[width=\textwidth]{\autofig{preco_simple_f1_1.pdf}}
    \vfill \centering
    \textcolor{green}{Mention Detection Entropy}
}
\only<2>{
    \includegraphics[width=\textwidth]{\autofig{preco_simple_f1_2.pdf}}
    \vfill \centering
    \textcolor{Orchid}{Mention Clustering Entropy}
}
\only<3>{
    \includegraphics[width=\textwidth]{\autofig{preco_simple_f1_3.pdf}}
    \vfill \centering
    \textcolor{red}{Conditional Entropy}
}
\only<4>{
    \includegraphics[width=\textwidth]{\autofig{preco_simple_f1_4.pdf}}
    \vfill \centering
    \textcolor{SkyBlue}{Joint Entropy}
}
\only<5>{
    \includegraphics[width=\textwidth]{\autofig{preco_simple_f1_5.pdf}}
    \vfill \centering
    \textcolor{Gray}{Random}
}
\only<6>{
    \includegraphics[width=\textwidth]{\autofig{preco_simple_f1_6.pdf}}
    \vfill \centering
    \textcolor{NavyBlue}{Random Entity Mentions}
}
%\only<7>{
    %\includegraphics[width=\textwidth]{\autofig{preco_simple_f1_6.pdf}}
%}
\column{0.4\textwidth}
\small
    \begin{itemize}
    \item
    Test Avg. F1 on PreCo
    \item For each cycle, we simulate labeling
    fifty spans from one document
%\onslide<7->
%\item \textbf{Ment-ent}, \textbf{clust-ent}, and \textbf{joint-ent} are effective while \textbf{random} performs worst
    \end{itemize}
\end{columns}
\end{frame}

\begin{frame}{Ment-ent, Clust-ent, Joint-ent are Effective but Random
    Performs Worst!}
\centering
\includegraphics[width=0.6\textwidth]{\autofig{preco_simple_f1_6.pdf}}
\end{frame}

% section 4.1.1
%\begin{frame}{Distribution of Sampled Span Types}
%\begin{columns}[c]
%\column{0.55\textwidth}
    %\includegraphics[width=\textwidth]{\autofig{preco_samples.pdf}}
%\column{0.45\textwidth}
%Cumulative counts of entities, non-entities,
%pronouns, and singletons sampled for each strategy
%over first four cycles of the PreCo simulation
%\end{columns}
%\end{frame}
