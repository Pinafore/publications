\section{Posterior Regularizers to Inject Knowledge}
\label{sec:post_reg}

Improving topic models often requires modeling additional information:
word correlations~\cite{boyd-graber-2007-tm-wsd}, document
labels~\cite{mcauliffe-2008-slda}, or document
links~\cite{chang-2010-rtm}.
%
Likewise, linking multilingual topics requires an infusion of
knowledge, either in the form of word translations~\cite{Hu-2014-ptlda}, paired documents~\cite{mimno-2009-plda},
or embeddings~\cite{das-2015-glda,ammar-2016-uw-embed}.
\wycomment{It looks like no work is done on linking multilingual topics using word embeddings. Shall we cite multilingual word embedding and monolingual Gaussian LDA? 
\jbgcomment{For camera ready, better to add citations wherever relevant}}
However, increasing the complexity of the model often comes at a cost:
inference becomes more complicated.  Posterior regularization becomes
an effective middle ground.  It's relatively flexible but able to model many
forms of metadata.

\newcite{yang-2015-knowledge} specifically apply posterior
regularization to Gibbs sampling~\cite{heinrich-09} for
vanilla topic models that sample only token topic
assignments~$\bm{z}$.
For each prior knowledge source~$m$ in the knowledge set~$M$, they
create a potential function $f_m (\bm{z}, \bm{w}, \bm{d})$ of topic assignments~$\bm{z}$,
words~$\bm{w}$, and/or documents~$\bm{d}$.
Higher $f_m (\bm{z}, \bm{w}, \bm{d})$ indicates better consistency
with~$m$ at the current state.  The posterior regularizer is
\begin{equation}\label{equ:post_reg}
\Psi \left( \bm{z}, \bm{w}, M \right) = \prod_{m \in M} \exp \left( f_m (\bm{z}, \bm{w}, \bm{d}) \right),
\end{equation}
which they include in the collapsed Gibbs sampling equation
as:\par\nobreak
\begin{align}
\Pr ( &z_{d,n}=k \,|\, \bm{z_{-d,n}}, w_{d,n}=v, \bm{w_{-d,n}} ) \notag\\
\propto & \explain{\textsc{lda} Sampling}{ \left( N_{d,k}^{-d,n}+\alpha \right) \frac{N_{k,v}^{-d,n}+\beta}{N_{k,\cdot}^{-d,n}+V\beta}} \notag\\
& \explain{Posterior Regularizer}{\Psi \left( z_{d,n}=k, \bm{z_{-d,n}}, \bm{w}, M \right)}, \label{equ:gibbs_post_reg}
\end{align}
where $N_{d,k}$ is the number of tokens in document~$d$ assigned to
topic~$k$; $N_{k,v}$ is the number of times that word~$v$ is assigned
to topic~$k$; and $\cdot$ denotes marginal counts; $^{-d,n}$ means
that the count excludes the token~\cite{heinrich-09,resnik-2010-gibbs}.\jbgcomment{Now that we have more time, ease into Gibbs sampling a little bit more.  Give intuition about what sampling step does, how the additional term changes things with an example (beyond just saying ``consistent with''.}

In this formulation, the potential functions shape Gibbs sampling
inference: topic assignments are more likely when they are consistent
with the prior knowledge included in the potential functions.
This brings significant flexibility in the expression of the potential
function $f_m (\bm{z}, \bm{w}, \bm{d})$.
The expression is not restricted to probabilistic distributions,
exponential family, or conjugacy.  \jbgcomment{``conjugacy'' isn't specific enough.  Need to say ``conjugate priors''}
It can be expressed flexibly using the combinations of \emph{any} of
the values from topic assignments, words, and/or documents. 
\jbgcomment{This is a little too much; you could make the computation
  too expensive to be done tractibly.  Would be good to reign this in
  a little bit.}
Moreover, changing the expressions of potential functions does not
change the main structure of the topic model or require full
re-derivation of the Gibbs sampling equation, which allows more
flexible experimentation.

For our \mtm, we encode the topic link weights and the translation
pairs' topic distributions into the posterior regularizer. Thus we can
convey prior knowledge across languages and improve the topic models
for the two languages.
